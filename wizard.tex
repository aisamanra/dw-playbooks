\documentclass[8pt]{extarticle}
\usepackage[dvipsnames]{xcolor}
\usepackage{hyperref}
\usepackage{lmodern}
\usepackage{amssymb,amsmath}
\usepackage{ifxetex,ifluatex}
\usepackage{anyfontsize}
\usepackage[percent]{overpic}
\usepackage[margin=0.4in]{geometry}
\usepackage{multicol}
\setlength{\columnsep}{0.05cm}
\usepackage[T1]{fontenc}
\usepackage[utf8]{inputenc}
\usepackage{fontspec} % For loading fonts
\usepackage{titlesec}

\setmainfont{PT Serif}
\newfontfamily\headingfont[]{Metamorphous}
\titleformat*{\section}{\LARGE\headingfont}
\titleformat*{\subsection}{\Large\headingfont}

\newenvironment{amove}[1]
{\Checkbox{6pt}\ {\color{MoveBlue}$\diamond$\headingfont #1}\begin{quote}
}
{\end{quote}
}

\newenvironment{optfeature}[2][]
{\Checkbox{6pt}\ {\headingfont #2}\hfill\textit{#1}\phantom{asdf}\begin{quote}
}
{\end{quote}
}

\newenvironment{aspell}[2]
{\Checkbox{6pt}\ {\headingfont\ \spell{#1}}\hfill\textit{#2}\phantom{asdf}\begin{quote}
}
{\end{quote}
}

\newenvironment{fragment}[1]
{\begin{quote}{\headingfont #1}\begin{quote}
}
{\end{quote}\end{quote}
}

\newcommand{\subheader}[1]{\large\headingfont #1}

\newenvironment{basicmove}[1]
{\begin{quote}{\color{MoveBlue}$\diamond$\headingfont #1}\begin{quote}
}
{\end{quote}\end{quote}
}

\makeatletter

\newcommand{\choicelabel}[1]{
  {\hss\llap{\Checkbox{6pt}}}
}
\newcommand{\choicelabeldef}{
  \@gobble{choicelabeldef}
}

\newenvironment{choices}
{
  \itemize
  \let\makelabel\choicelabel
  \let\@itemlabel\choicelabeldef
}
{\enditemize
}
\makeatother

\newcommand{\choice}{\Checkbox{6pt} }

\pagestyle{empty}
\IfFileExists{upquote.sty}{\usepackage{upquote}}{}
% use microtype if available
\IfFileExists{microtype.sty}{%
\usepackage[]{microtype}
\UseMicrotypeSet[protrusion]{basicmath} % disable protrusion for tt fonts
}{}
\PassOptionsToPackage{hyphens}{url} % url is loaded by hyperref

\makeatother
% Scale images if necessary, so that they will not overflow the page
% margins by default, and it is still possible to overwrite the defaults
% using explicit options in \includegraphics[width, height, ...]{}
\setkeys{Gin}{width=\maxwidth,height=\maxheight,keepaspectratio}
\IfFileExists{parskip.sty}{%
\usepackage{parskip}
}{% else
\setlength{\parindent}{0pt}
\setlength{\parskip}{6pt plus 2pt minus 1pt}
}
\setlength{\emergencystretch}{3em}  % prevent overfull lines
\providecommand{\tightlist}{%
  \setlength{\itemsep}{0pt}\setlength{\parskip}{0pt}}
\setcounter{secnumdepth}{0}
% Redefines (sub)paragraphs to behave more like sections
\ifx\paragraph\undefined\else
\let\oldparagraph\paragraph
\renewcommand{\paragraph}[1]{\oldparagraph{#1}\mbox{}}
\fi
\ifx\subparagraph\undefined\else
\let\oldsubparagraph\subparagraph
\renewcommand{\subparagraph}[1]{\oldsubparagraph{#1}\mbox{}}
\fi

% set default figure placement to htbp
\makeatletter
\def\fps@figure{htbp}
\makeatother

\setlength{\multicolsep}{6.0pt plus 2.0pt minus 1.5pt}% 50% of original values

\date{}

\usepackage{etoolbox}
\patchcmd{\quote}{\rightmargin}{\leftmargin 1em \rightmargin}{}{}

\usepackage{tikz}
\newcommand{\Checkbox}[1]{\tikz{\path[draw=black] (0,0) rectangle (#1,#1);}}

\newcommand{\pbClass}[1]{\newcommand{\Class}{#1}}
\newcommand{\pbBaseHP}[1]{\newcommand{\BaseHP}{#1}}
\newcommand{\pbDamage}[1]{\newcommand{\Damage}{#1}}
\newcommand{\Look}{}
\newcommand{\Names}{}
\makeatletter
\newcommand{\pbLook}[1]{\g@addto@macro\Look{\par#1}}
\newcommand{\pbNames}[2]{\g@addto@macro\Names{\par\hangindent=0.2in#1: #2}}
\makeatother

\newcommand{\leftbanner}[1]{
  \begin{overpic}[width=3.1in,height=0.45in]{assets/short_left.png}
\put (2,4) {\fontsize{16}{40}\selectfont \textcolor{white}{\headingfont #1}}
\end{overpic}
}

\newcommand{\rightbanner}[1]{
  \begin{overpic}[width=4.4in,height=0.45in]{assets/long_right.png}
\put (5,4) {\fontsize{16}{40}\selectfont \textcolor{white}{\headingfont #1}}
\end{overpic}
}

\newcommand{\gearbanner}{
\begin{overpic}[width=7.47986in,height=0.40945in]{assets/templateEquip.png}
\put (3,2) {\fontsize{16}{40}\selectfont \textcolor{white}{\headingfont Gear}}
\end{overpic}
}

\newcommand{\topbanner}[1]{
  \begin{overpic}[width=7.47986in,height=1.0in]{assets/templateRuleHeader.png}
  \put (1,2) {\fontsize{32}{40}\selectfont\headingfont \textcolor{white}{#1}}
\end{overpic}
}

\newcommand{\widebanner}[1]{
  \begin{overpic}[width=7.47986in,height=1.0in]{assets/templateThinHeader.png}
  \put (1,1) {\fontsize{16}{40}\selectfont\headingfont \textcolor{white}{#1}}
\end{overpic}
}

\newcommand{\charbanner}{
  \begin{overpic}[width=8.008in,height=3.0in]{assets/charsheet_upper.png}
  % names
  \put(1, 30) {\parbox{4.3in}{\fontsize{12}{12}\Names}}
  % look
  \put(59, 30) {\parbox{3in}{\fontsize{12}{12}\Look}}

  % some stats: damage...
  \put (25,4) {\makebox[0pt]{\fontsize{18}{10}\selectfont\headingfont \textcolor{black}{D\Damage{}}}}
  % max HP...
  \put (89,6) {\fontsize{6}{8}\selectfont\headingfont \textcolor{white}{Your max HP is}}
  % and Constitution
  \put (89,4.6) {\fontsize{6}{8}\selectfont\headingfont \textcolor{white}{\BaseHP{} + Constitution}}
\end{overpic}
}

\newcommand{\charlower}{
  \vfill\null
  \begin{overpic}[width=7.47986in,height=1.0in]{assets/charsheet_lower.png}
  \put (10,1) {\fontsize{32}{40}\selectfont\headingfont \textcolor{white}{The \Class}}
\end{overpic}
}

\definecolor{CondRed}{RGB}{153,51,51}
\definecolor{MoveBlue}{RGB}{51,102,153}
\definecolor{SpellPurp}{RGB}{153,51,102}
\definecolor{TagGreen}{RGB}{51,153,102}

\newcommand{\condition}[1]{\textbf{\color{CondRed} #1}}
\newcommand{\move}[1]{{\color{MoveBlue}$\diamond$#1}}
\newcommand{\spell}[1]{{\color{SpellPurp}$\star$#1}}
\newcommand{\itag}[1]{\textit{\color{TagGreen}#1}}
\newcommand{\ntag}[2]{\textit{\color{TagGreen}#1 #2}}

% specific tags
\newcommand{\weight}[1]{\ntag{#1}{weight}}
\newcommand{\damage}[1]{\ntag{#1}{damage}}
\newcommand{\armor}[1]{\ntag{#1}{armor}}
\newcommand{\armorForward}[1]{\ntag{#1}{armor} forward}
\newcommand{\uses}[1]{\ntag{#1}{uses}}
\newcommand{\ammo}[1]{\ntag{#1}{ammo}}

\newcommand{\hexes}[1]{\textit{#1 hexes}}
\newcommand{\forward}[1]{#1 forward}
\newcommand{\ongoing}[1]{#1 ongoing}
\newcommand{\yourLoad}[1]{Your Load is \textbf{#1+STR}}

\newcommand{\onSuccess}{\textbf{On a 10+}}
\newcommand{\onPlainSuccess}{\textbf{On a 10--11}}
\newcommand{\onMassiveSuccess}{\textbf{On a 12+}}
\newcommand{\onPartial}{\textbf{On a 7--9}}
\newcommand{\onHit}{\textbf{On a 7+}}
\newcommand{\onMiss}{\textbf{On a miss}}

\newcommand{\moveReplaces}[1]{\textbf{Replaces}: \move{#1}}
\newcommand{\moveRequires}[1]{\textbf{Requires}: \move{#1}}
\newcommand{\moveRequiresLst}[1]{\textbf{Requires}: #1}

\newcommand{\onMassiveSuccessFor}[1]{%
  When you use \move{#1} and \textbf{roll a 12+}}
\newcommand{\onPlainSuccessFor}[1]{%
  When you use \move{#1} and \textbf{roll a 10--11}}
\newcommand{\onPartialFor}[1]{%
  When you use \move{#1} and \textbf{roll a 7+}}

\newcommand{\advancesigil}{$\triangleright$}
\newcommand{\firstAdvances}{\advancesigil When you \textbf{gain a
    level from 2--5}, choose from these moves.}
\newcommand{\secondAdvances}{\advancesigil When you \textbf{gain a
    level from 6--10}, choose from these moves or the level 2--5
  moves.}

\newcommand{\blank}{\underline{\phantom{mountain}}}
\newcommand{\directive}[1]{\textbf{#1}}

\openup -0.2em

\pbClass{Wizard}
\pbBaseHP{4}
\pbDamage{4}

\pbNames{Elf}{Galadiir, Fenfaril, Lilliastre, Phirosalle, Enkirash,
  Halwyr}
\pbNames{Human}{Avon, Morgan, Rath, Ysolde, Ovid, Vitus,
  Aldara, Xeno, Uri}

\pbLook{Haunted Eyes, Sharp Eyes, or Crazy Eyes}
\pbLook{Styled Hair, Wild Hair, or Pointed Hat}
\pbLook{Worn Robes, Stylish Robes, or Strange Robes}
\pbLook{Pudgy Body, Creepy Body, or Thin Body}

\begin{document}
\openup -0.2em

\charbanner

\begin{minipage}[t]{3.2in}
\leftbanner{Folk}

\begin{optfeature}{Elf}
  Magic is as natural as breath to you. \spell{Detect Magic} is a
  cantrip for you.
\end{optfeature}

\begin{optfeature}{Human}
  Choose one Cleric spell. You can cast it as if it was a Wizard
  spell.
\end{optfeature}

\ 

\leftbanner{Alignment}

\begin{optfeature}{Good}
  Use magic to directly aid another.
\end{optfeature}

\begin{optfeature}{Neutral}
  Discover something about a magical mystery.
\end{optfeature}

\begin{optfeature}{Evil}
  Use magic to cause terror and fear.
\end{optfeature}

\ 

\leftbanner{Bonds}


\vfill\null
\end{minipage}
\begin{minipage}[t]{4.6in}


\rightbanner{Starting Moves}

\begin{basicmove}{Spellbook}
  You have mastered several spells and inscribed them in your
  spellbook. You start out with three first level spells in your
  spellbook as well as the cantrips. Whenever you gain a level, you
  add a new spell of your level or lower to your spellbook. You
  spellbook is 1 weight.
\end{basicmove}
\

\begin{basicmove}{Prepare Spells}
  When you \condition{spend uninterrupted time (an hour or so) in
    quiet contemplation of your spellbook}, you:

  \begin{itemize}
  \item Lose any spells you already have prepared
  \item Prepare new spells of your choice from your spellbook whose
    total levels don’t exceed your own level+1.
  \item Prepare your cantrips which never count against your limit.
  \end{itemize}
\end{basicmove}
\

\begin{basicmove}{Cast a Spell (INT)}
  When you \condition{release a spell you’ve prepared},
  roll +INT. \onSuccess, the spell is successfully cast and you do not
  forget the spell—you may cast it again later. \onPartial, the spell
  is cast, but choose one:

  \begin{itemize}
  \item You draw unwelcome attention or put yourself in a spot. The GM
    will tell you how.
  \item The spell disturbs the fabric of reality
    as it is cast—take \ongoing{-1} to cast a spell until the next time
    you Prepare Spells.
  \item After it is cast, the spell is forgotten. You cannot cast the
    spell again until you prepare spells.
  \end{itemize}

  Note that maintaining spells with ongoing effects will sometimes
  cause a penalty to your roll to cast a spell.
\end{basicmove}
\

\begin{basicmove}{Spell Defense}
  You may end any ongoing spell immediately and use the energy of its
  dissipation to deflect an oncoming attack. The spell ends and you
  subtract its level from the damage done to you.
\end{basicmove}
\

\begin{basicmove}{Ritual}
  When you \condition{draw on a place of power to create a magical
    effect}, tell the GM what you’re trying to achieve. Ritual effects
  are always possible, but the GM will give you one to four of the
  following conditions:

  \begin{itemize}
  \item It’s going to take days/weeks/months.
  \item First you must                       .
  \item You’ll need help from                       .
  \item It will require a lot of money
  \item The best you can do is a lesser version, unreliable and limited
  \item You and your allies will risk danger from                       .
  \item You’ll have to disenchant \blank to do it.
  \end{itemize}
\end{basicmove}


\vfill\null
\end{minipage}
\charlower

\clearpage

\gearbanner

\begin{multicols}{2}

  \yourLoad{7}. You start with your spellbook (\weight{1}) and dungeon
  rations (\uses{5}, \weight{1}). Choose your defenses:

  \begin{choices}
  \item Leather armor (\armor{1}, \weight{1})
  \item Bag of books (\uses{5}, \weight{2}) and 3 healing potions
  \end{choices}

  Choose your weapon:

  \begin{choices}
  \item Dagger (\itag{hand}, \weight{1})
  \item Staff (\itag{close}, \itag{two-handed}, \weight{1})
  \end{choices}

  Choose one:

  \begin{choices}
  \item Healing potion (\weight{0})
  \item 3 antitoxins (\weight{0})
  \end{choices}

\columnbreak

\

\end{multicols}

\widebanner{Advanced Moves}

\begin{multicols}{2}
  \firstAdvances

\begin{amove}{Prodigy}
  Choose a spell. You prepare that spell as if it were one level
  lower.
\end{amove}

\begin{amove}{Empowered Magic}
  When you \move{Cast a Spell}, on a 10+ you have the option of
  choosing from the 7-9 list. If you do, you may choose one of these
  as well:

  \begin{itemize}
  \item The spell’s effects are maximized
  \item The spell’s targets are doubled
  \end{itemize}
\end{amove}


\begin{amove}{Fount of Knowledge}
  When you \move{Spout Lore} about something no one else has any clue
  about, take +1.
\end{amove}

\begin{amove}{Know-It-All}
  When \condition{another player’s character comes to you for advice
    and you tell them what you think is best}, they get \forward{+1}
  when following your advice and you mark experience if they do.
\end{amove}

\begin{amove}{Expanded Spellbook}
  Add a new spell from the spell list of any class to your spellbook.
\end{amove}

\begin{amove}{Enchanter}
  When you \condition{have time and safety with a magic item} you may
  ask the GM what it does, the GM will answer you truthfully.
\end{amove}

\begin{amove}{Logical}
  When you \condition{use strict deduction to analyze your
    surroundings}, you can discern realities with INT instead of WIS.
\end{amove}

\begin{amove}{Arcane Ward}
  As long as you have at least one prepared spell of first level or
  higher, you have \armor{+2}.
\end{amove}

\begin{amove}{Counterspell}
  When you \condition{attempt to counter an arcane spell that will
    otherwise affect you}, stake one of your prepared spells on the
  defense and roll +INT. \onSuccess, the spell is countered and has no
  effect on you. \onPartial, the spell is countered and you forget the
  spell you staked. Your counterspell protects only you; if the
  countered spell has other targets they get its effects.
\end{amove}

\begin{amove}{Quick Study}
  When you \condition{see the effects of an arcane spell}, ask the GM
  the name of the spell and its effects. You take +1 when acting on
  the answers.
\end{amove}


\vfill\null
\columnbreak

\secondAdvances

\begin{amove}{Master}
  \moveRequires{Prodigy}

  Choose one spell in addition to the one you picked for prodigy. You
  prepare that spell as if it were one level lower.
\end{amove}

\begin{amove}{Greater Empowered Magic}
  \moveReplaces{Empowered Magic}

  \onPlainSuccessFor{Cast A Spell}, you have the option of choosing
  from the 7-9 list. If you do, you may choose one of these effects as
  well. \onMassiveSuccess, you get to choose one of these effects for
  free:

  \begin{itemize}
  \item The spell’s effects are doubled
  \item The spell’s targets are doubled
  \end{itemize}
\end{amove}

\begin{amove}{Enchanter’s Soul}
  \moveRequires{Enchanter}

  When you \condition{have time and safety with a magic item in a
    place of power} you can empower that item so that the next time
  you use it its effects are amplified, the GM will tell you exactly
  how.
\end{amove}

\begin{amove}{Highly Logical}
  \moveReplaces{Logical}

  When you \condition{use strict deduction to analyze your
    surroundings}, you can discern realities with INT instead of
  WIS. \onMassiveSuccess, you get to ask the GM any three questions,
  not limited by the list.
\end{amove}

\begin{amove}{Arcane Armor}
\moveReplaces{Arcane Ward}

  As long as you have at least one prepared spell of first level or
  higher, you have \armor{+4}.
\end{amove}

\begin{amove}{Protective Counter}
\moveRequires{Counterspell}

  When \condition{an ally within sight of you is affected by an arcane
    spell}, you can counter it as if it affected you. If the spell
  affects multiple allies you must counter for each ally separately.
\end{amove}

\begin{amove}{Ethereal Tether}
  When you \condition{have time with a willing or helpless subject}
  you can craft an ethereal tether with them. You perceive what they
  perceive and can discern realities about someone tethered to you or
  their surroundings no matter the distance. Someone willingly
  tethered to you can communicate with you over the tether as if you
  were in the room with them.
\end{amove}

\begin{amove}{Mystical Puppet Strings}
  When you \condition{use magic to control a person’s actions} they
  have no memory of what you had them do and bear you no ill will.
\end{amove}

\begin{amove}{Spell Augmentation}
  When you \condition{deal damage to a creature} you can shunt a
  spell’s energy into them—end one of your ongoing spells and add the
  spell’s level to the damage dealt.
\end{amove}

\begin{amove}{Self-Powered}
  When you \condition{have time, arcane materials, and a safe space},
  you can create your own place of power. Describe to the GM what kind
  of power it is and how you’re binding it to this place, the GM will
  tell you one kind of creature that will have an interest in your
  workings.
\end{amove}


\vfill\null
\end{multicols}
\clearpage


\topbanner{Wizard Spells}

\


\widebanner{Cantrips}
\begin{multicols}{2}

  \begin{quote}
    You prepare all of your cantrips every time you prepare spells
    without having to select them or count them toward your allotment
    of spells.
  \end{quote}

  \begin{aspell}{Light}{}
    An item you touch glows with arcane light, about as bright as a
    torch. It gives off no heat or sound and requires no fuel, but it
    is otherwise like a mundane torch.  You have complete control of
    the color of the flame. The spell lasts as long as it is in your
    presence.
  \end{aspell}

\vfill\null
\columnbreak

  \begin{aspell}{Unseen Servant}{}
    You conjure a simple invisible construct that can do nothing but
    carry items. It has Load 3 and carries anything you hand to it. It
    cannot pick up items on its own and can only carry those you give
    to it. Items carried by an unseen servant appear to float in the
    air a few paces behind you. An unseen servant that takes damage or
    leaves your presence is immediately dispelled, dropping any items
    it carried.
  \end{aspell}

  \begin{aspell}{Prestidigitation}{}
    You perform minor tricks of true magic. If you touch an item as
    part of the casting you can make cosmetic changes to it: clean it,
    soil it, cool it, warm it, flavor it, or change its color. If you
    cast the spell without touching an item you can instead create
    minor illusions no bigger than yourself. Prestidigitation
    illusions are crude and clearly illusions—they won’t fool anyone,
    but they might entertain them.
  \end{aspell}

\vfill\null
\end{multicols}

\widebanner{First Level Spells}
\begin{multicols}{2}

  \begin{aspell}{Contact Spirits}{Summoning}
    Name the spirit you wish to contact (or leave it to the GM). You
    pull that creature through the planes, just close enough to speak
    to you. It is bound to answer any one question you ask to the best
    of its ability.
  \end{aspell}

  \begin{aspell}{Detect Magic}{Divination}
    One of your senses is briefly attuned to magic. The GM will tell
    you what here is magical.
  \end{aspell}

  \begin{aspell}{Telepathy}{Divination Ongoing}
    You form a telepathic bond with a single person you touch,
    enabling you to converse with that person through your
    thoughts. You can only have one telepathic bond at a time.
  \end{aspell}

\vfill\null
\columnbreak

\begin{aspell}{Charm Person}{Enchantment Ongoing}
  The person (not beast or monster) you touch while casting this spell
  counts you as a friend until they take damage or you prove
  otherwise.
\end{aspell}

\begin{aspell}{Invisibility}{Illusion Ongoing}
  Touch an ally: nobody can see them. They’re invisible! The spell
  persists until the target attacks or you dismiss the effect. While
  the spell is ongoing you can’t cast a spell.
\end{aspell}

\begin{aspell}{Magic Missile}{Evocation}
  Projectiles of pure magic spring from your fingers. Deal 2d4 damage
  to one target.
\end{aspell}

\begin{aspell}{Alarm}{}
  Walk a wide circle as you cast this spell. Until you prepare spells
  again your magic will alert you if a creature crosses that
  circle. Even if you are asleep, the spell will shake you from your
  slumber.
\end{aspell}


\vfill\null
\end{multicols}

\widebanner{Third Level Spells}
\begin{multicols}{2}

  \begin{aspell}{Dispel Magic}{}
    Choose a spell or magic effect in your presence: this spell rips
    it apart. Lesser spells are ended, powerful magic is just reduced
    or dampened so long as you are nearby.
  \end{aspell}

  \begin{aspell}{Visions Through Time}{Divination}
    Cast this spell and gaze into a reflective surface to see into the
    depths of time. The GM will reveal the details of a grim portent
    to you—a bleak event that will come to pass without your
    intervention. They’ll tell you something useful about how you can
    interfere with the grim portent’s dark outcomes. Rare is the
    portent that claims “You’ll live happily ever after.” Sorry.
  \end{aspell}

  \begin{aspell}{Fireball}{Evocation}
    You evoke a mighty ball of flame that envelops your target and
    everyone nearby, inflicting 2d6 damage which ignores armor.
  \end{aspell}

  \vfill\null
  \columnbreak

  \begin{aspell}{Mimic}{Ongoing}
    You take the form of someone you touch while casting this
    spell. Your physical characteristics match theirs exactly but your
    behavior may not. This change persists until you take damage or
    choose to return to your own form. While this spell is ongoing you
    lose access to all your Wizard moves.
  \end{aspell}

  \begin{aspell}{Mirror Image}{Illusion}
    You create an illusory image of yourself. When you are attacked,
    roll a d6. On a 4, 5, or 6 the attack hits the illusion instead,
    the image then dissipates and the spell ends.
  \end{aspell}

  \begin{aspell}{Sleep}{Enchantment}
    1d4 enemies you can see of the GM’s choice fall asleep. Only
    creatures capable of sleeping are affected. They awake as normal:
    loud noises, jolts, pain.
  \end{aspell}

\vfill\null
\end{multicols}

\clearpage

\topbanner{Wizard Spells}

\widebanner{Fifth Level Spells}
\begin{multicols}{2}

  \begin{aspell}{Cage}{Evocation Ongoing}
    The target is held in a cage of magical force. Nothing can get in
    or out of the cage. The cage remains until you cast another spell
    or dismiss it. While the spell is ongoing, the caged creature can
    hear your thoughts and you cannot leave sight of the cage.
  \end{aspell}

  \begin{aspell}{Contact Other Plane}{Divination}
    You send a request to another plane. Specify who or what you’d
    like to contact by location, type of creature, name, or title. You
    open a two-way communication with that creature. Your
    communication can be cut off at any time by you or the creature
    you contacted.
  \end{aspell}

  \begin{aspell}{Polymorph}{Enchantment}
    Your touch reshapes a creature entirely, they stay in the form you
    craft until you cast a spell. Describe the new shape you craft,
    including any stat changes, significant adaptations, or major
    weaknesses. The GM will then tell you one or more of these:
    \begin{itemize}
    \item The form will be unstable and temporary
    \item The creature’s mind will be altered as well
    \item The form has an unintended benefit or weakness
    \end{itemize}
  \end{aspell}

  \vfill\null
  \columnbreak

  \begin{aspell}{Summon Monster}{Summoning Ongoing}
    A monster appears and aids you as best it can. Treat it as your
    character, but with access to only the basic moves. It has +1
    modifier for all stats, 1 HP, and uses your damage dice. The
    monster also gets your choice of 1d6 of these traits:
    \begin{itemize}
    \item It has +2 instead of +1 to one stat
    \item It’s not reckless
    \item It does 1d8 damage
    \item Its bond to your plane is strong: +2 HP for each level you have
    \item It has some useful adaptation
    \end{itemize}

    The GM will tell you the type of monster you get based on the
    traits you select.  The creature remains on this plane until it
    dies or you dismiss it. While the spell is ongoing you take -1 to
    cast a spell.
  \end{aspell}


\vfill\null
\end{multicols}

\widebanner{Seventh Level Spells}
\begin{multicols}{2}

  \begin{aspell}{Dominate}{Enchantment Ongoing}
    Your touch pushes your mind into someone else’s. You gain 1d4
    hold. Spend one hold to make the target take one of these actions:
    \begin{itemize}
    \item Speak a few words of your choice
    \item Give you something they hold
    \item Make a concerted attack on a target of your choice
    \item Truthfully answer one question
    \end{itemize}

    If you run out of hold the spell ends. If the target takes damage
    you lose 1 hold.  While the spell is ongoing you cannot cast a
    spell.
  \end{aspell}

  \begin{aspell}{True Seeing}{Divination Ongoing}
    You see all things as they truly are. This effect persists until
    you tell a lie or dismiss the spell. While this spell is ongoing
    you take -1 to cast a spell.
  \end{aspell}


  \begin{aspell}{Shadow Walk}{Illusion}
    The shadows you target with this spell become a portal for you and
    your allies.  Name a location, describing it with a number of
    words up to your level. Stepping through the portal deposits you
    and any allies present when you cast the spell at the location you
    described. The portal may only be used once by each ally.
  \end{aspell}

  \vfill\null
  \columnbreak

  \begin{aspell}{Contingency}{Evocation}
    Choose a 5th level or lower spell you know. Describe a trigger
    condition using a number of words equal to your level. The chosen
    spell is held until you choose to unleash it or the trigger
    condition is met, whichever happens first. You don’t have to roll
    for the held spell, it just takes effect. You may only have a
    single contingent spell held at a time; if you cast Contingency
    while you have a held spell, the new held spell replaces the old
    one.
  \end{aspell}

  \begin{aspell}{Cloudkill}{Summoning Ongoing}
    A cloud of fog drifts into this realm from beyond the Black Gates
    of Death, filling the immediate area. Whenever a creature in the
    area takes damage it takes an additional, separate 1d6 damage
    which ignores armor. This spell persists so long as you can see
    the affected area, or until you dismiss it.
  \end{aspell}

\vfill\null
\end{multicols}

\widebanner{Ninth Level Spells}
\begin{multicols}{2}

  \begin{aspell}{Antipathy}{Enchantment Ongoing}
    Choose a target and describe a type of creature or an
    alignment. Creatures of the specified type or alignment cannot
    come within sight of the target. If a creature of the specified
    type does find itself within sight of the target, it immediately
    flees. This effect continues until you leave the target’s presence
    or you dismiss the spell. While the spell is ongoing you take -1
    to cast a spell.
  \end{aspell}

  \begin{aspell}{Alert}{Divination}
    Describe an event. The GM will tell you when that event occurs, no
    matter where you are or how far away the event is. If you choose,
    you can view the location of the event as though you were there in
    person. You can only have one Alert active at a time.
  \end{aspell}

  \vfill\null
  \columnbreak

  \begin{aspell}{Soul Gem}{}
    You trap the soul of a dying creature within a gem. The trapped
    creature is aware of its imprisonment but can still be manipulated
    through spells, parley, and other effects. All moves against the
    trapped creature are at +1. You can free the soul at any time but
    it can never be recaptured once freed.
  \end{aspell}

  \begin{aspell}{Shelter}{Evocation Ongoing}
    You create a structure out of pure magical power. It can be as
    large as a castle or as small as a hut, but is impervious to all
    non-magical damage. The structure endures until you leave it or
    you end the spell.
  \end{aspell}

  \begin{aspell}{Perfect Summons}{Summoning}
    You teleport a creature to your presence. Name a creature or give
    a short description of a type of creature. If you named a
    creature, that creature appears before you. If you described a
    type of creature, a creature of that type appears before you.
  \end{aspell}

\vfill\null
\end{multicols}

\end{document}
