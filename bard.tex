\documentclass[8pt]{extarticle}

\usepackage[dvipsnames]{xcolor}
\usepackage{hyperref}
\usepackage{lmodern}
\usepackage{amssymb,amsmath}
\usepackage{ifxetex,ifluatex}
\usepackage{anyfontsize}
\usepackage[percent]{overpic}
\usepackage[margin=0.4in]{geometry}
\usepackage{multicol}
\setlength{\columnsep}{0.05cm}
\usepackage[T1]{fontenc}
\usepackage[utf8]{inputenc}
\usepackage{fontspec} % For loading fonts
\usepackage{titlesec}

\setmainfont{PT Serif}
\newfontfamily\headingfont[]{Metamorphous}
\titleformat*{\section}{\LARGE\headingfont}
\titleformat*{\subsection}{\Large\headingfont}

\newenvironment{amove}[1]
{\Checkbox{6pt}\ {\color{MoveBlue}$\diamond$\headingfont #1}\begin{quote}
}
{\end{quote}
}

\newenvironment{optfeature}[2][]
{\Checkbox{6pt}\ {\headingfont #2}\hfill\textit{#1}\phantom{asdf}\begin{quote}
}
{\end{quote}
}

\newenvironment{aspell}[2]
{\Checkbox{6pt}\ {\headingfont\ \spell{#1}}\hfill\textit{#2}\phantom{asdf}\begin{quote}
}
{\end{quote}
}

\newenvironment{fragment}[1]
{\begin{quote}{\headingfont #1}\begin{quote}
}
{\end{quote}\end{quote}
}

\newcommand{\subheader}[1]{\large\headingfont #1}

\newenvironment{basicmove}[1]
{\begin{quote}{\color{MoveBlue}$\diamond$\headingfont #1}\begin{quote}
}
{\end{quote}\end{quote}
}

\makeatletter

\newcommand{\choicelabel}[1]{
  {\hss\llap{\Checkbox{6pt}}}
}
\newcommand{\choicelabeldef}{
  \@gobble{choicelabeldef}
}

\newenvironment{choices}
{
  \itemize
  \let\makelabel\choicelabel
  \let\@itemlabel\choicelabeldef
}
{\enditemize
}
\makeatother

\newcommand{\choice}{\Checkbox{6pt} }

\pagestyle{empty}
\IfFileExists{upquote.sty}{\usepackage{upquote}}{}
% use microtype if available
\IfFileExists{microtype.sty}{%
\usepackage[]{microtype}
\UseMicrotypeSet[protrusion]{basicmath} % disable protrusion for tt fonts
}{}
\PassOptionsToPackage{hyphens}{url} % url is loaded by hyperref

\makeatother
% Scale images if necessary, so that they will not overflow the page
% margins by default, and it is still possible to overwrite the defaults
% using explicit options in \includegraphics[width, height, ...]{}
\setkeys{Gin}{width=\maxwidth,height=\maxheight,keepaspectratio}
\IfFileExists{parskip.sty}{%
\usepackage{parskip}
}{% else
\setlength{\parindent}{0pt}
\setlength{\parskip}{6pt plus 2pt minus 1pt}
}
\setlength{\emergencystretch}{3em}  % prevent overfull lines
\providecommand{\tightlist}{%
  \setlength{\itemsep}{0pt}\setlength{\parskip}{0pt}}
\setcounter{secnumdepth}{0}
% Redefines (sub)paragraphs to behave more like sections
\ifx\paragraph\undefined\else
\let\oldparagraph\paragraph
\renewcommand{\paragraph}[1]{\oldparagraph{#1}\mbox{}}
\fi
\ifx\subparagraph\undefined\else
\let\oldsubparagraph\subparagraph
\renewcommand{\subparagraph}[1]{\oldsubparagraph{#1}\mbox{}}
\fi

% set default figure placement to htbp
\makeatletter
\def\fps@figure{htbp}
\makeatother

\setlength{\multicolsep}{6.0pt plus 2.0pt minus 1.5pt}% 50% of original values

\date{}

\usepackage{etoolbox}
\patchcmd{\quote}{\rightmargin}{\leftmargin 1em \rightmargin}{}{}

\usepackage{tikz}
\newcommand{\Checkbox}[1]{\tikz{\path[draw=black] (0,0) rectangle (#1,#1);}}

\newcommand{\pbClass}[1]{\newcommand{\Class}{#1}}
\newcommand{\pbBaseHP}[1]{\newcommand{\BaseHP}{#1}}
\newcommand{\pbDamage}[1]{\newcommand{\Damage}{#1}}
\newcommand{\Look}{}
\newcommand{\Names}{}
\makeatletter
\newcommand{\pbLook}[1]{\g@addto@macro\Look{\par#1}}
\newcommand{\pbNames}[2]{\g@addto@macro\Names{\par\hangindent=0.2in#1: #2}}
\makeatother

\newcommand{\leftbanner}[1]{
  \begin{overpic}[width=3.1in,height=0.45in]{assets/short_left.png}
\put (2,4) {\fontsize{16}{40}\selectfont \textcolor{white}{\headingfont #1}}
\end{overpic}
}

\newcommand{\rightbanner}[1]{
  \begin{overpic}[width=4.4in,height=0.45in]{assets/long_right.png}
\put (5,4) {\fontsize{16}{40}\selectfont \textcolor{white}{\headingfont #1}}
\end{overpic}
}

\newcommand{\gearbanner}{
\begin{overpic}[width=7.47986in,height=0.40945in]{assets/templateEquip.png}
\put (3,2) {\fontsize{16}{40}\selectfont \textcolor{white}{\headingfont Gear}}
\end{overpic}
}

\newcommand{\topbanner}[1]{
  \begin{overpic}[width=7.47986in,height=1.0in]{assets/templateRuleHeader.png}
  \put (1,2) {\fontsize{32}{40}\selectfont\headingfont \textcolor{white}{#1}}
\end{overpic}
}

\newcommand{\widebanner}[1]{
  \begin{overpic}[width=7.47986in,height=1.0in]{assets/templateThinHeader.png}
  \put (1,1) {\fontsize{16}{40}\selectfont\headingfont \textcolor{white}{#1}}
\end{overpic}
}

\newcommand{\charbanner}{
  \begin{overpic}[width=8.008in,height=3.0in]{assets/charsheet_upper.png}
  % names
  \put(1, 30) {\parbox{4.3in}{\fontsize{12}{12}\Names}}
  % look
  \put(59, 30) {\parbox{3in}{\fontsize{12}{12}\Look}}

  % some stats: damage...
  \put (25,4) {\makebox[0pt]{\fontsize{18}{10}\selectfont\headingfont \textcolor{black}{D\Damage{}}}}
  % max HP...
  \put (89,6) {\fontsize{6}{8}\selectfont\headingfont \textcolor{white}{Your max HP is}}
  % and Constitution
  \put (89,4.6) {\fontsize{6}{8}\selectfont\headingfont \textcolor{white}{\BaseHP{} + Constitution}}
\end{overpic}
}

\newcommand{\charlower}{
  \vfill\null
  \begin{overpic}[width=7.47986in,height=1.0in]{assets/charsheet_lower.png}
  \put (10,1) {\fontsize{32}{40}\selectfont\headingfont \textcolor{white}{The \Class}}
\end{overpic}
}

\definecolor{CondRed}{RGB}{153,51,51}
\definecolor{MoveBlue}{RGB}{51,102,153}
\definecolor{SpellPurp}{RGB}{153,51,102}
\definecolor{TagGreen}{RGB}{51,153,102}

\newcommand{\condition}[1]{\textbf{\color{CondRed} #1}}
\newcommand{\move}[1]{{\color{MoveBlue}$\diamond$#1}}
\newcommand{\spell}[1]{{\color{SpellPurp}$\star$#1}}
\newcommand{\itag}[1]{\textit{\color{TagGreen}#1}}
\newcommand{\ntag}[2]{\textit{\color{TagGreen}#1 #2}}

% specific tags
\newcommand{\weight}[1]{\ntag{#1}{weight}}
\newcommand{\damage}[1]{\ntag{#1}{damage}}
\newcommand{\armor}[1]{\ntag{#1}{armor}}
\newcommand{\armorForward}[1]{\ntag{#1}{armor} forward}
\newcommand{\uses}[1]{\ntag{#1}{uses}}
\newcommand{\ammo}[1]{\ntag{#1}{ammo}}

\newcommand{\hexes}[1]{\textit{#1 hexes}}
\newcommand{\forward}[1]{#1 forward}
\newcommand{\ongoing}[1]{#1 ongoing}
\newcommand{\yourLoad}[1]{Your Load is \textbf{#1+STR}}

\newcommand{\onSuccess}{\textbf{On a 10+}}
\newcommand{\onPlainSuccess}{\textbf{On a 10--11}}
\newcommand{\onMassiveSuccess}{\textbf{On a 12+}}
\newcommand{\onPartial}{\textbf{On a 7--9}}
\newcommand{\onHit}{\textbf{On a 7+}}
\newcommand{\onMiss}{\textbf{On a miss}}

\newcommand{\moveReplaces}[1]{\textbf{Replaces}: \move{#1}}
\newcommand{\moveRequires}[1]{\textbf{Requires}: \move{#1}}
\newcommand{\moveRequiresLst}[1]{\textbf{Requires}: #1}

\newcommand{\onMassiveSuccessFor}[1]{%
  When you use \move{#1} and \textbf{roll a 12+}}
\newcommand{\onPlainSuccessFor}[1]{%
  When you use \move{#1} and \textbf{roll a 10--11}}
\newcommand{\onPartialFor}[1]{%
  When you use \move{#1} and \textbf{roll a 7+}}

\newcommand{\advancesigil}{$\triangleright$}
\newcommand{\firstAdvances}{\advancesigil When you \textbf{gain a
    level from 2--5}, choose from these moves.}
\newcommand{\secondAdvances}{\advancesigil When you \textbf{gain a
    level from 6--10}, choose from these moves or the level 2--5
  moves.}

\newcommand{\blank}{\underline{\phantom{mountain}}}
\newcommand{\directive}[1]{\textbf{#1}}

\openup -0.2em

\pbClass{Bard}
\pbBaseHP{6}
\pbDamage{6}

\pbNames{Elf}{Astrafel, Daelwyn, Feliana, Damarra, Sistranalle,
  Pendrell, Melliandre, Dagoliir}

\pbNames{Human}{Baldric, Leena, Dunwick, Willem, Edwyn, Florian,
  Seraphine, Quorra, Charlotte, Lily, Ramonde, Cassandra}

\pbLook{Knowing Eyes, Fiery Eyes, or Joyous Eyes}
\pbLook{Fancy Hair, Wild Hair, or Stylish Cap}
\pbLook{Finery, Traveling Clothes, or Poor Clothes}
\pbLook{Fit Body, Well-fed Body, or Thin Body}


\begin{document}
\openup -0.2em

\charbanner

\begin{minipage}[t]{3.2in}
  \leftbanner{Folk}


\begin{optfeature}{Elf}
  When you enter an important location (your call) you can ask the GM
  for one fact from the history of that location.
\end{optfeature}

\begin{optfeature}{Human}
  When you first enter a civilized settlement someone who respects the
  custom of hospitality to minstrels will take you in as their guest.
\end{optfeature}

\

\leftbanner{Alignment}

\begin{optfeature}{Good}
  Perform your art to aid someone else.
\end{optfeature}

\begin{optfeature}{Neutral}
  Avoid a conflict or defuse a tense situation.
\end{optfeature}

\begin{optfeature}{Chaotic}
  Spur others to significant and unplanned decisive action.
\end{optfeature}

\

\leftbanner{Bonds}
\vfill\null

\end{minipage}
\begin{minipage}[t]{4.6in}

\rightbanner{Starting Moves}

\begin{basicmove}{Arcane Art}
  When you \condition{weave a performance into a basic spell}, choose
  an ally and an effect:

  \begin{itemize}
  \item Heal 1d8 damage
  \item \forward{+1d4} to damage
  \item Their mind is shaken clear of one enchantment
  \item The next time someone successfully assists the target with
    aid, they get +2 instead of +1
  \end{itemize}

  Then roll +CHA. \onSuccess, the ally gets the selected
  effect. \onPartial, your spell still works, but you draw unwanted
  attention or your magic reverberates to other targets affecting them
  as well, GM’s choice.
\end{basicmove}
\

\begin{basicmove}{Bardic Lore}
  Choose an area of expertise:

  \begin{choices}
  \item Spells and Magicks
  \item The Dead and Undead
  \item Grand Histories of the Known World
  \item A Bestiary of Creatures Unusual
  \item The Planar Spheres
  \item Legends of Heroes Past
  \item Gods and Their Servants
  \end{choices}

  When you \condition{first encounter an important creature, location,
    or item (your call) covered by your bardic lore} you can ask the
  GM any one question about it; the GM will answer truthfully. The GM
  may then ask you what tale, song, or legend you heard that
  information in.
\end{basicmove}
\

\begin{basicmove}{Charming and Open}
  When you \condition{speak frankly with someone}, you can ask their
  player a question from the list below. They must answer it
  truthfully, then they may ask you a question from the list (which
  you must answer truthfully).

  \begin{itemize}
  \item Whom do you serve?
  \item What do you wish I would do?
  \item How can I get you to \blank?
  \item What are you really feeling right now?
  \item What do you most desire?
  \end{itemize}

\end{basicmove}
\

\begin{basicmove}{A Port in the Storm}
    When you \condition{arrive at a civilized settlement spoken of in
      lore or song}, tell the GM something you've heard about the
    place. They’ll tell you how it’s changed since the Shattering.
\end{basicmove}


\vfill\null
\end{minipage}

\charlower
\clearpage

\gearbanner

\begin{multicols}{2}

\begin{quote}
  \yourLoad{9}. You have dungeon rations (\ntag{5}{uses},
  \ntag{1}{weight}). Choose one instrument, all are \ntag{0}{weight}
  for you:

  \begin{choices}
  \item Your father’s mandolin, repaired
  \item A fine lute, a gift from a noble
  \item The pipes with which you courted your first love
  \item A stolen horn
  \item A fiddle, never before played
  \item A songbook in a forgotten tongue
  \end{choices}

\end{quote}
\vfill\null

\ 

\columnbreak
\begin{quote}
  Choose your clothing:

  \begin{choices}
  \item Leather armor (\armor{1}, \weight{1})
  \item Ostentatious clothes (\weight{0})
  \end{choices}

  Choose your armament:

  \begin{choices}
  \item Dueling rapier (\itag{close}, \itag{precise}, \ntag{2}{weight})
  \item Worn bow (\itag{near}, \ntag{2}{weight}), bundle of arrows
    (\ntag{3}{ammo}, \itag{1 weight}), and short sword (\itag{close},
    \ntag{1}{weight})
  \end{choices}

  Choose one:

  \begin{choices}
  \item Adventuring gear (\ntag{1}{weight})
  \item Bandages (\ntag{0}{weight})
  \item Halfling pipeleaf (\ntag{0}{weight})
  \item 3 coins
  \end{choices}

\end{quote}

\end{multicols}

\widebanner{Advanced Moves}

\begin{multicols}{2}
\firstAdvances

\begin{amove}{Healing Song}
  When you \condition{heal with \move{Arcane Art}}, you heal +1d8
  damage.
\end{amove}

\begin{amove}{Vicious Cacophony}
  When you \condition{grant bonus damage with \move{Arcane Art}}, you
  grant an extra +1d4 damage.
\end{amove}

\begin{amove}{It Goes To Eleven}
  When you \condition{unleash a crazed performance} (a righteous lute
  solo or mighty brass blast, maybe) choose a target who can hear you
  and roll +CHA. \onSuccess, the target attacks their nearest ally in
  range. \onPartial, they attack their nearest ally, but you also draw
  their attention and ire.
\end{amove}


\begin{amove}{Metal Hurlant}
  When you \condition{shout with great force or play a shattering
    note} choose a target and roll +CON. \onSuccess, the target takes
  1d10 damage and is deafened for a few minutes. \onPartial, you still
  damage your target, but it’s out of control: the GM will choose an
  additional target nearby.
\end{amove}


\begin{amove}{A Little Help From My Friends}
  When you \condition{successfully aid someone} you take \forward{+1} as
  well.
\end{amove}


\begin{amove}{Eldritch Tones}
  Your \move{Arcane Art} is strong, allowing you to choose two effects
  instead of one.
\end{amove}


\begin{amove}{Duelist’s Parry}
  When you \move{Hack and Slash}, you take \armorForward{+1}.
\end{amove}

\begin{amove}{Bamboozle}
  \onPartialFor{Parley}, you also take \forward{+1} with them.
\end{amove}


\begin{amove}{Multiclass Dabbler}
  Get one move from another class. Treat your level as one lower for
  choosing the move.
\end{amove}


\begin{amove}{Multiclass Initiate}
  Get one move from another class. Treat your level as one lower for
  choosing the move.
\end{amove}

\vfill\null
\columnbreak

\secondAdvances

\begin{amove}{Healing Chorus}
\moveReplaces{Healing Song}

  When you \condition{heal with \move{Arcane Art}}, you heal +2d8 damage.
\end{amove}


\begin{amove}{Vicious Blast}
\moveReplaces{Vicious Cacophony}

  When you \condition{grant bonus damage with \move{Arcane Art}}, you
  grant an extra +2d4 damage.
\end{amove}


\begin{amove}{Unforgettable Face}
  When you \condition{meet someone you’ve met before} (your call)
  after some time apart you take \forward{+1} against them.
\end{amove}


\begin{amove}{Reputation}
  When you \condition{first meet someone who’s heard songs about you},
  roll +CHA. \onSuccess, tell the GM two things they’ve heard about
  you. \onPartial, tell the GM one thing they’ve heard, and the GM
  tells you one thing.
\end{amove}


\begin{amove}{Eldritch Chord}
\moveReplaces{Eldritch Tones}

  When you use \move{Arcane Art}, you choose two effects. You also get to
  choose one of those effects to double.
\end{amove}


\begin{amove}{An Ear For Magic}
  When you \condition{hear an enemy cast a spell} the GM will tell you
  the name of the spell and its effects. Take \forward{+1} when acting
  on the answers.
\end{amove}


\begin{amove}{Devious}
  When you use \move{Charming and Open} you may also ask “How are you
  vulnerable to me?” Your subject may not ask this question of you.
\end{amove}


\begin{amove}{Duelist’s Block}
\moveReplaces{Duelist’s Parry}

  When you \move{Hack and Slash}, you take \armorForward{+2}.
\end{amove}


\begin{amove}{Con}
  \moveReplaces{Bamboozle}

  \onPartialFor{Parley}, you also take \forward{+1} with them and get
  to ask their player one question which they must answer truthfully.
\end{amove}


\begin{amove}{Multiclass Master}
  Get one move from another class. Treat your level as one lower for
  choosing the move.
\end{amove}

\vfill\null
\end{multicols}

\end{document}
