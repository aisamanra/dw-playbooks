\documentclass[8pt]{extarticle}

\usepackage[dvipsnames]{xcolor}
\usepackage{hyperref}
\usepackage{lmodern}
\usepackage{amssymb,amsmath}
\usepackage{ifxetex,ifluatex}
\usepackage{anyfontsize}
\usepackage[percent]{overpic}
\usepackage[margin=0.4in]{geometry}
\usepackage{multicol}
\setlength{\columnsep}{0.05cm}
\usepackage[T1]{fontenc}
\usepackage[utf8]{inputenc}
\usepackage{fontspec} % For loading fonts
\usepackage{titlesec}

\setmainfont{PT Serif}
\newfontfamily\headingfont[]{Metamorphous}
\titleformat*{\section}{\LARGE\headingfont}
\titleformat*{\subsection}{\Large\headingfont}

\newenvironment{amove}[1]
{\Checkbox{6pt}\ {\color{MoveBlue}$\diamond$\headingfont #1}\begin{quote}
}
{\end{quote}
}

\newenvironment{optfeature}[2][]
{\Checkbox{6pt}\ {\headingfont #2}\hfill\textit{#1}\phantom{asdf}\begin{quote}
}
{\end{quote}
}

\newenvironment{aspell}[2]
{\Checkbox{6pt}\ {\headingfont\ \spell{#1}}\hfill\textit{#2}\phantom{asdf}\begin{quote}
}
{\end{quote}
}

\newenvironment{fragment}[1]
{\begin{quote}{\headingfont #1}\begin{quote}
}
{\end{quote}\end{quote}
}

\newcommand{\subheader}[1]{\large\headingfont #1}

\newenvironment{basicmove}[1]
{\begin{quote}{\color{MoveBlue}$\diamond$\headingfont #1}\begin{quote}
}
{\end{quote}\end{quote}
}

\makeatletter

\newcommand{\choicelabel}[1]{
  {\hss\llap{\Checkbox{6pt}}}
}
\newcommand{\choicelabeldef}{
  \@gobble{choicelabeldef}
}

\newenvironment{choices}
{
  \itemize
  \let\makelabel\choicelabel
  \let\@itemlabel\choicelabeldef
}
{\enditemize
}
\makeatother

\newcommand{\choice}{\Checkbox{6pt} }

\pagestyle{empty}
\IfFileExists{upquote.sty}{\usepackage{upquote}}{}
% use microtype if available
\IfFileExists{microtype.sty}{%
\usepackage[]{microtype}
\UseMicrotypeSet[protrusion]{basicmath} % disable protrusion for tt fonts
}{}
\PassOptionsToPackage{hyphens}{url} % url is loaded by hyperref

\makeatother
% Scale images if necessary, so that they will not overflow the page
% margins by default, and it is still possible to overwrite the defaults
% using explicit options in \includegraphics[width, height, ...]{}
\setkeys{Gin}{width=\maxwidth,height=\maxheight,keepaspectratio}
\IfFileExists{parskip.sty}{%
\usepackage{parskip}
}{% else
\setlength{\parindent}{0pt}
\setlength{\parskip}{6pt plus 2pt minus 1pt}
}
\setlength{\emergencystretch}{3em}  % prevent overfull lines
\providecommand{\tightlist}{%
  \setlength{\itemsep}{0pt}\setlength{\parskip}{0pt}}
\setcounter{secnumdepth}{0}
% Redefines (sub)paragraphs to behave more like sections
\ifx\paragraph\undefined\else
\let\oldparagraph\paragraph
\renewcommand{\paragraph}[1]{\oldparagraph{#1}\mbox{}}
\fi
\ifx\subparagraph\undefined\else
\let\oldsubparagraph\subparagraph
\renewcommand{\subparagraph}[1]{\oldsubparagraph{#1}\mbox{}}
\fi

% set default figure placement to htbp
\makeatletter
\def\fps@figure{htbp}
\makeatother

\setlength{\multicolsep}{6.0pt plus 2.0pt minus 1.5pt}% 50% of original values

\date{}

\usepackage{etoolbox}
\patchcmd{\quote}{\rightmargin}{\leftmargin 1em \rightmargin}{}{}

\usepackage{tikz}
\newcommand{\Checkbox}[1]{\tikz{\path[draw=black] (0,0) rectangle (#1,#1);}}

\newcommand{\pbClass}[1]{\newcommand{\Class}{#1}}
\newcommand{\pbBaseHP}[1]{\newcommand{\BaseHP}{#1}}
\newcommand{\pbDamage}[1]{\newcommand{\Damage}{#1}}
\newcommand{\Look}{}
\newcommand{\Names}{}
\makeatletter
\newcommand{\pbLook}[1]{\g@addto@macro\Look{\par#1}}
\newcommand{\pbNames}[2]{\g@addto@macro\Names{\par\hangindent=0.2in#1: #2}}
\makeatother

\newcommand{\leftbanner}[1]{
  \begin{overpic}[width=3.1in,height=0.45in]{assets/short_left.png}
\put (2,4) {\fontsize{16}{40}\selectfont \textcolor{white}{\headingfont #1}}
\end{overpic}
}

\newcommand{\rightbanner}[1]{
  \begin{overpic}[width=4.4in,height=0.45in]{assets/long_right.png}
\put (5,4) {\fontsize{16}{40}\selectfont \textcolor{white}{\headingfont #1}}
\end{overpic}
}

\newcommand{\gearbanner}{
\begin{overpic}[width=7.47986in,height=0.40945in]{assets/templateEquip.png}
\put (3,2) {\fontsize{16}{40}\selectfont \textcolor{white}{\headingfont Gear}}
\end{overpic}
}

\newcommand{\topbanner}[1]{
  \begin{overpic}[width=7.47986in,height=1.0in]{assets/templateRuleHeader.png}
  \put (1,2) {\fontsize{32}{40}\selectfont\headingfont \textcolor{white}{#1}}
\end{overpic}
}

\newcommand{\widebanner}[1]{
  \begin{overpic}[width=7.47986in,height=1.0in]{assets/templateThinHeader.png}
  \put (1,1) {\fontsize{16}{40}\selectfont\headingfont \textcolor{white}{#1}}
\end{overpic}
}

\newcommand{\charbanner}{
  \begin{overpic}[width=8.008in,height=3.0in]{assets/charsheet_upper.png}
  % names
  \put(1, 30) {\parbox{4.3in}{\fontsize{12}{12}\Names}}
  % look
  \put(59, 30) {\parbox{3in}{\fontsize{12}{12}\Look}}

  % some stats: damage...
  \put (25,4) {\makebox[0pt]{\fontsize{18}{10}\selectfont\headingfont \textcolor{black}{D\Damage{}}}}
  % max HP...
  \put (89,6) {\fontsize{6}{8}\selectfont\headingfont \textcolor{white}{Your max HP is}}
  % and Constitution
  \put (89,4.6) {\fontsize{6}{8}\selectfont\headingfont \textcolor{white}{\BaseHP{} + Constitution}}
\end{overpic}
}

\newcommand{\charlower}{
  \vfill\null
  \begin{overpic}[width=7.47986in,height=1.0in]{assets/charsheet_lower.png}
  \put (10,1) {\fontsize{32}{40}\selectfont\headingfont \textcolor{white}{The \Class}}
\end{overpic}
}

\definecolor{CondRed}{RGB}{153,51,51}
\definecolor{MoveBlue}{RGB}{51,102,153}
\definecolor{SpellPurp}{RGB}{153,51,102}
\definecolor{TagGreen}{RGB}{51,153,102}

\newcommand{\condition}[1]{\textbf{\color{CondRed} #1}}
\newcommand{\move}[1]{{\color{MoveBlue}$\diamond$#1}}
\newcommand{\spell}[1]{{\color{SpellPurp}$\star$#1}}
\newcommand{\itag}[1]{\textit{\color{TagGreen}#1}}
\newcommand{\ntag}[2]{\textit{\color{TagGreen}#1 #2}}

% specific tags
\newcommand{\weight}[1]{\ntag{#1}{weight}}
\newcommand{\damage}[1]{\ntag{#1}{damage}}
\newcommand{\armor}[1]{\ntag{#1}{armor}}
\newcommand{\armorForward}[1]{\ntag{#1}{armor} forward}
\newcommand{\uses}[1]{\ntag{#1}{uses}}
\newcommand{\ammo}[1]{\ntag{#1}{ammo}}

\newcommand{\hexes}[1]{\textit{#1 hexes}}
\newcommand{\forward}[1]{#1 forward}
\newcommand{\ongoing}[1]{#1 ongoing}
\newcommand{\yourLoad}[1]{Your Load is \textbf{#1+STR}}

\newcommand{\onSuccess}{\textbf{On a 10+}}
\newcommand{\onPlainSuccess}{\textbf{On a 10--11}}
\newcommand{\onMassiveSuccess}{\textbf{On a 12+}}
\newcommand{\onPartial}{\textbf{On a 7--9}}
\newcommand{\onHit}{\textbf{On a 7+}}
\newcommand{\onMiss}{\textbf{On a miss}}

\newcommand{\moveReplaces}[1]{\textbf{Replaces}: \move{#1}}
\newcommand{\moveRequires}[1]{\textbf{Requires}: \move{#1}}
\newcommand{\moveRequiresLst}[1]{\textbf{Requires}: #1}

\newcommand{\onMassiveSuccessFor}[1]{%
  When you use \move{#1} and \textbf{roll a 12+}}
\newcommand{\onPlainSuccessFor}[1]{%
  When you use \move{#1} and \textbf{roll a 10--11}}
\newcommand{\onPartialFor}[1]{%
  When you use \move{#1} and \textbf{roll a 7+}}

\newcommand{\advancesigil}{$\triangleright$}
\newcommand{\firstAdvances}{\advancesigil When you \textbf{gain a
    level from 2--5}, choose from these moves.}
\newcommand{\secondAdvances}{\advancesigil When you \textbf{gain a
    level from 6--10}, choose from these moves or the level 2--5
  moves.}

\newcommand{\blank}{\underline{\phantom{mountain}}}
\newcommand{\directive}[1]{\textbf{#1}}

\openup -0.2em

\pbClass{Cleric}
\pbBaseHP{8}
\pbDamage{6}

\pbNames{Dwarf}{Durga, Aelfar, Gerda, Rurgosh, Bjorn, Drummond, Helga,
  Siggrun, Freya}

\pbNames{Human}{Wesley, Brinton, Jon, Sara, Hawthorn, Elise, Clarke,
  Lenore, Piotr, Dahlia, Carmine}

\pbLook{Kind Eyes, Sharp Eyes, or Sad Eyes}
\pbLook{Tonsure, Strange Hair, or Bald}
\pbLook{Flowing Robes, Habit, or Common Garb}
\pbLook{Thin Body, Knobby Body, or Flabby Body}


\begin{document}
\openup -0.2em

\charbanner

\begin{minipage}[t]{3.2in}
\leftbanner{Folk}

\begin{optfeature}{Dwarf}
  You are one with stone. When you \move{Commune} you are also granted
  a special version of \spell{Words of the Unspeaking} as a rote which
  only works on stone.
\end{optfeature}

\begin{optfeature}{Human}
  Your faith is diverse. Choose one Wizard spell. You can cast and be
  granted that spell as if it was a Cleric spell.
\end{optfeature}

\ 

\leftbanner{Alignment}

\begin{optfeature}{Good}
  Endanger yourself to heal another.
\end{optfeature}

\begin{optfeature}{Lawful}
  Endanger yourself following the precepts of your church or god.
\end{optfeature}

\begin{optfeature}{Evil}
  Harm another to prove the superiority of your church or god.
\end{optfeature}


\ 

\leftbanner{Bonds}


\vfill\null
\end{minipage}
\begin{minipage}[t]{4.6in}
\rightbanner{Starting Moves}

\begin{basicmove}{Deity}
  You serve and worship some deity or power which grants you
  spells. Give your god a name (maybe Helferth, Sucellus, Zorica or
  Krugon the Bleak) and choose your deity’s domain:

% this is one of the grosser ones: it'd be nice to have a generic way
% of solving this
\begin{multicols}{2}
  \begin{choices}
  \item Healing and Restoration
  \item Bloody Conquest
  \item Civilization
  \end{choices}
  \columnbreak
  \begin{choices}
  \item Knowledge and Secrets
  \item The Downtrodden
  \item What Lies Beneath
  \end{choices}
\end{multicols}

Choose one precept of your religion:
\begin{choices}
\item Your religion preaches the sanctity of suffering, add Petition:
  Suffering
\item Your religion is cultish and insular, add Petition: Gaining
  Secrets
\item Your religion has important sacrificial rites, add Petition:
  Offering
\item Your religion believes in trial by combat, add Petition:
  Personal Victory
\end{choices}
\end{basicmove}

\begin{basicmove}{Divine Guidance}
  When you \condition{petition your deity according to the precept of
    your religion}, you are granted some useful knowledge or boon
  related to your deity’s domain. The GM will tell you what.
\end{basicmove}

\begin{basicmove}{Turn Undead}
  When you \condition{hold your holy symbol aloft and call on your
    deity for protection}, roll +WIS. \onHit, so long as you continue
  to pray and brandish your holy symbol, no undead may come within
  reach of you. \onSuccess, you also momentarily daze intelligent
  undead and cause mindless undead to flee. Aggression breaks the
  effects and they are able to act as normal. Intelligent undead may
  still find ways to harry you from afar. They’re clever like that.
\end{basicmove}

\begin{basicmove}{Commune}
  When you \condition{spend uninterrupted time (an hour or so) in
    quiet communion with your deity}, you:

  \begin{itemize}
  \item Lose any spells already granted to you.
  \item Are granted new spells of your choice whose total levels don’t
    exceed your own level+1, and none of which is a higher level than
    your own level.
  \item Prepare all of your rotes, which never count against your
    limit.
  \end{itemize}
\end{basicmove}

\begin{basicmove}{Cast a Spell}

  When you \condition{unleash a spell granted to you by your deity},
  roll +WIS. \onSuccess, the spell is successfully cast and your
  deity does not revoke the spell, so you may cast it
  again. \onPartial, the spell is cast, but choose one:

  \begin{itemize}
  \item You draw unwelcome attention or put yourself in a spot. The GM
    will tell you how.
  \item Your casting distances you from your deity---take \ongoing{-1}
    to cast a spell until the next time you \move{Commune}.
  \item After you cast it, the spell is revoked by your deity. You
    cannot cast the spell again until you \move{Commune} and have it
    granted to you.
  \end{itemize}

  Note that maintaining spells with ongoing effects will sometimes
  cause a penalty to your roll to cast a spell.
\end{basicmove}

\vfill\null
\end{minipage}

\charlower
\clearpage

\gearbanner

\begin{multicols}{2}

  \begin{quote}
    \yourLoad{10}. You carry dungeon rations (\ntag{5}{uses},
    \ntag{1}{weight}) and some symbol of the divine, describe it
    (\ntag{0}{weight}). Choose your defenses:

    \begin{quote}
      \Checkbox{6pt} Chainmail (\armor{1}, \ntag{1}{weight})

    \Checkbox{6pt} Shield (\ntag{+1}{armor}, \ntag{2}{weight})
    \end{quote}

    Choose your armament:
    \begin{quote}
    \Checkbox{6pt} Warhammer (\itag{close}, \ntag{1}{weight})

    \Checkbox{6pt} Mace (\itag{close}, \ntag{1}{weight})

    \Checkbox{6pt} Staff (\itag{close}, \itag{two-handed},
    \ntag{1}{weight}) and bandages (\ntag{0}{weight})
    \end{quote}

    Choose one:
    \begin{quote}
      \Checkbox{6pt} Adventuring gear (\ntag{1}{weight}) and dungeon
      rations (\ntag{5}{uses}, \ntag{1}{weight})

      \Checkbox{6pt} Healing potion (\ntag{0}{weight})
    \end{quote}
\end{quote}

\columnbreak

\ 

\end{multicols}

\widebanner{Advanced Moves}

\begin{multicols}{2}
\firstAdvances

\begin{amove}{Chosen One}
  Choose one spell. You are granted that spell as if it was one level
  lower.
\end{amove}

\begin{amove}{Invigorate}
  When you \condition{heal someone} they take \forward{+2} to their
  damage.
\end{amove}

\begin{amove}{The Scales of Life and Death}
  When someone takes their \move{Last Breath} in your presence, they
  take +1 to the roll.
\end{amove}

\begin{amove}{Serenity}
  When you \move{Cast A Spell} you ignore the first -1 penalty from
  ongoing spells.
\end{amove}

\begin{amove}{First Aid}
  \spell{Cure Light Wounds} is a rote for you, and therefore doesn’t
  count against your limit of granted spells.
\end{amove}

\begin{amove}{Divine Intervention}
  When you \move{Commune} you get 1 hold and lose any hold you already
  had. Spend that hold when you or an ally takes damage to call on
  your deity, they intervene with an appropriate manifestation (a
  sudden gust of wind, a lucky slip, a burst of light) and negate the
  damage.
\end{amove}

\begin{amove}{Penitent}
  When you \condition{take damage and embrace the pain}, you may take
  \damage{+1d4} (ignoring armor). If you do, take \forward{+1} to cast
  a spell.
\end{amove}

\begin{amove}{Empower}
  When you \move{Cast A Spell}, on a 10+ you have the option of
  choosing from the 7–9 list. If you do, you may choose one of these
  effects as well:

  \begin{itemize}
  \item The spell’s effects are doubled
  \item The spell’s targets are doubled
  \end{itemize}
\end{amove}

\begin{amove}{Orison for Guidance}
  When you \condition{sacrifice something of value to your deity and
    pray for guidance}, your deity tells you what it would have you
  do. If you do it, mark experience.
\end{amove}

\begin{amove}{Divine Protection}
  When you \condition{wear no armor or shield} you get \armor{2}.
\end{amove}

\begin{amove}{Devoted Healer}
  When you \condition{heal someone else of damage}, add your level to
  the amount of damage healed.
\end{amove}


\vfill\null
\columnbreak

\secondAdvances

\begin{amove}{Anointed}
\moveRequires{Chosen One}

  Choose one spell in addition to the one you picked for \move{Chosen
    One}. You are granted that spell as if it was one level lower.
\end{amove}

\begin{amove}{Apotheosis}
  The first time you spend time in prayer as appropriate to your god
  after taking this move, choose a feature associated with your deity
  (rending claws, wings of sapphire feathers, an all-seeing third eye,
  etc.). When you emerge from prayer, you permanently gain that
  physical feature.
\end{amove}

\begin{amove}{Reaper}
  When you \condition{take time after a conflict to dedicate your
    victory to your deity and deal with the dead}, take \forward{+1}.
\end{amove}

\begin{amove}{Providence}
\moveReplaces{Serenity}

  You ignore the -1 penalty from two spells you maintain.
\end{amove}

\begin{amove}{Greater First Aid}
\moveRequires{First Aid}

  \spell{Cure Moderate Wounds} is a rote for you, and therefore
  doesn’t count against your limit of granted spells.
\end{amove}

\begin{amove}{Divine Invincibility}
\moveReplaces{Divine Intervention}

  When you \move{Commune} you gain 2 hold and lose any hold you
  already had. Spend that hold when you or an ally takes damage to
  call on your deity, who intervenes with an appropriate manifestation
  (a sudden gust of wind, a lucky slip, a burst of light) and negates
  the damage.
\end{amove}

\begin{amove}{Martyr}
\moveReplaces{Penitent}

When you \condition{take damage and embrace the pain}, you may take
+1d4 damage (ignoring armor). If you do, take \forward{+1} to cast a
spell and add your level to any damage done or healed by the spell.
\end{amove}

\begin{amove}{Divine Armor}
\moveReplaces{Divine Protection}

  When you \condition{wear no armor or shield} you get \armor{3}.
\end{amove}

\begin{amove}{Greater Empower}
  \moveReplaces{Empower}

  \onPlainSuccessFor{Cast A Spell}, you have the option of choosing
  from the 7–9 list. If you do, you may choose one of these effects as
  well. \onMassiveSuccess, you get to choose one of these effects for
  free.

  \begin{itemize}
  \item The spell’s effects are doubled
  \item The spell’s targets are doubled
  \end{itemize}
\end{amove}


\begin{amove}{Multiclass Dabbler}
  Get one move from another class. Treat your level as one lower for
  choosing the move.
\end{amove}

\vfill\null
\end{multicols}
\clearpage

\topbanner{Cleric Spells}

\

\widebanner{Rotes}
\begin{multicols}{2}
  \begin{quote}
    Every time you \move{Commune}, you gain access to all of your
    rotes without having to select them or count them toward your
    allotment of spells.
  \end{quote}

  \begin{aspell}{Light}{}
    An item you touch glows with divine light, about as bright as a
    torch. It gives off no heat or sound and requires no fuel but is
    otherwise like a mundane torch.  You have complete control of the
    color of the flame. The spell lasts as long as it is in your
    presence.
  \end{aspell}

  \vfill\null
  \columnbreak

  \begin{aspell}{Sanctify}{}
    Food or water you hold in your hands while you cast this spell is
    consecrated by your deity. In addition to now being holy or
    unholy, the affected substance is purified of any mundane
    spoilage.
  \end{aspell}

  \begin{aspell}{Guidance}{}
    The symbol of your deity appears before you and gestures towards
    the direction or course of action your deity would have you take
    then disappears. The message is through gesture only; your
    communication through this spell is severely limited.
  \end{aspell}
\vfill\null
\end{multicols}

\widebanner{First Level Spells}

\begin{multicols}{2}
  \begin{aspell}{Bless}{Ongoing}
    Your deity smiles upon a combatant of your choice. They take
    \ongoing{+1} so long as battle continues and they stand and
    fight. While this spell is ongoing you take -1 to cast a spell.
  \end{aspell}

  \begin{aspell}{Cure Light Wounds}{}
    At your touch wounds scab and bones cease to ache. Heal an ally
    you touch of 1d8 damage.
  \end{aspell}

  \begin{aspell}{Detect Alignment}{}
    When you cast this spell choose an alignment: Good, Evil, Lawful,
    or Chaotic. One of your senses is briefly able to detect that
    alignment. The GM will tell you what here is of that alignment.
  \end{aspell}

  \begin{aspell}{Cause Fear}{Ongoing}
    Choose a target you can see and a nearby object. The target is
    afraid of the object so long as you maintain the spell. Their
    reaction is up to them: flee, panic, beg, fight. While this spell
    is ongoing you take -1 to cast a spell. You cannot target entities
    with less than animal intelligence (magical constructs, undead,
    automatons, and the like).
  \end{aspell}
  \vfill\null
  \columnbreak
  \begin{aspell}{Magic Weapon}{Ongoing}
    The weapon you hold while casting does +1d4 damage until you
    dismiss this spell. While this spell is ongoing you take -1 to
    cast a spell.
  \end{aspell}

  \begin{aspell}{Sanctuary}{}
    As you cast this spell, you walk the perimeter of an area,
    consecrating it to your deity. As long as you stay within that
    area you are alerted whenever someone acts with malice within the
    sanctuary (including entering with harmful intent).  Anyone who
    receives healing within a sanctuary heals +1d4 HP.
  \end{aspell}

  \begin{aspell}{Speak With Dead}{}
    A corpse converses with you briefly. It will answer any three
    questions you pose to it to the best of the knowledge it had in
    life and the knowledge it gained in death.
  \end{aspell}
  \vfill\null
\end{multicols}
\widebanner{Third Level Spells}
\begin{multicols}{2}
  \begin{aspell}{Animate Dead}{Ongoing}
    You invoke a hungry spirit to possess a recently-dead body and
    serve you. This creates a zombie that follows your orders to the
    best of its limited abilities. Treat the zombie as a character,
    but with access to only the basic moves. It has a +1 modifier for
    all stats and 1 HP. The zombie also gets your choice of 1d4 of
    these traits:
    \begin{itemize}
    \item It’s talented. Give one stat a +2 modifier.
    \item It’s durable. It has +2 HP for each level you have.
    \item It has a functioning brain and can complete complex tasks.
    \item It does not appear obviously dead, at least for a day or
      two.
    \end{itemize}
    The zombie lasts until it is destroyed by taking damage in excess
    of its HP, or until you end the spell. While this spell is ongoing
    you take -1 to cast a spell.
  \end{aspell}

  \begin{aspell}{Cure Moderate Wounds}{}
    You staunch bleeding and set bones through magic. Heal an ally you
    touch of 2d8 damage.
  \end{aspell}

  \begin{aspell}{Darkness}{Ongoing}
    Choose an area you can see: it’s filled with supernatural darkness
    and shadow.  While this spell is ongoing you take -1 to cast a
    spell.
  \end{aspell}
  \vfill\null
  \columnbreak

  \begin{aspell}{Resurrection}{}
    Tell the GM you would like to resurrect a corpse whose soul has
    not yet fully departed this world. Resurrection is always
    possible, but the GM will give you one or more (possibly all) of
    these conditions to fulfill:
    \begin{itemize}
    \item It’s going to take days/weeks/months
    \item You must get help from \blank
    \item It will require a lot of money
    \item You must sacrifice \blank\ to do it
    \end{itemize}
    The GM may, depending on the circumstances, allow you to resurrect
    the corpse now, with the understanding that the conditions must be
    met before it’s permanent, or require you to meet the conditions
    before the corpse is resurrected.
  \end{aspell}

  \begin{aspell}{Hold Person}{}
    Choose a person you can see. Until you cast a spell or leave their
    presence they cannot act except to speak. This effect ends
    immediately if the target takes damage from any source.
  \end{aspell}
  \vfill\null
\end{multicols}
\clearpage

\widebanner{Fifth Level Spells}
\begin{multicols}{2}
\begin{aspell}{Revelation}{}
  Your deity answers your prayers with a moment of perfect
  understanding. The GM will shed light on the current situation. When
  acting on the information, you take \forward{+1}.
\end{aspell}

\begin{aspell}{Cure Critical Wounds}{}
  Heal an ally you touch of 3d8 damage.
\end{aspell}

\begin{aspell}{Divination}{}
  Name a person, place, or thing you want to learn about. Your deity
  grants you visions of the target, as clear as if you were there.
\end{aspell}

\begin{aspell}{Contagion}{Ongoing}
  Choose a creature you can see. Until you end this spell, the target
  suffers from a disease of your choice. While this spell is ongoing
  you take -1 to cast a spell.
\end{aspell}
\vfill\null
\columnbreak

\begin{aspell}{Words of the Unspeaking}{}
  With a touch you speak to the spirits within things. The non-living
  object you touch answers three questions you pose, as best it can.
\end{aspell}

\begin{aspell}{True Seeing}{Ongoing}
  Your vision is opened to the true nature of everything you lay your
  eyes on. You pierce illusions and see things that have been
  hidden. The GM will describe the area before you ignoring any
  illusions and falsehoods, magical or otherwise.  While this spell is
  ongoing you take -1 to cast a spell.
\end{aspell}

\begin{aspell}{Trap Soul}{}
  You trap the soul of a dying creature within a gem. The trapped
  creature is aware of its imprisonment but can still be manipulated
  through spells, parley, and other effects. All moves against the
  trapped creature are at +1. You can free the soul at any time but it
  can never be recaptured once freed.
\end{aspell}

\vfill\null
\end{multicols}

\widebanner{Seventh Level Spells}

\begin{multicols}{2}
\begin{aspell}{Word of Recall}{}
  Choose a word. The first time after casting this spell that you
  speak the chosen word, you and any allies touching you when you cast
  the spell are immediately returned to the exact spot where you cast
  the spell. You can only maintain a single location; casting Word of
  Recall again before speaking the word replaces the earlier spell.
\end{aspell}

\begin{aspell}{Heal}{}
  Touch an ally and you may heal their damage a number of points up to
  your maximum HP.
\end{aspell}

\begin{aspell}{Harm}{}
  Touch an enemy and strike them with divine wrath—deal 2d8 damage to
  them and 1d6 damage to yourself. This damage ignores armor.
\end{aspell}
\vfill\null
\columnbreak

\begin{aspell}{Sever}{Ongoing}
  Choose an appendage on the target such as an arm, tentacle, or
  wing. The appendage is magically severed from their body, causing no
  damage but considerable pain. Missing an appendage may, for example,
  keep a winged creature from flying, or a bull from goring you on its
  horns. While this spell is ongoing you take -1 to cast a spell.
\end{aspell}

\begin{aspell}{Mark of Death}{}
  Choose a creature whose true name you know. This spell creates
  permanent runes on a target surface that will kill that creature,
  should they read them.
\end{aspell}

\begin{aspell}{Control Weather}{}
Pray for rain—or sun, wind, or snow. Within a day or so, your god will answer.
The weather will change according to your will and last a handful of days
\end{aspell}
\vfill\null
\end{multicols}

\widebanner{Ninth Level Spells}

\begin{multicols}{2}
\begin{aspell}{Storm of Vengeance}{}
  Your deity brings the unnatural weather of your choice to pass. Rain
  of blood or acid, clouds of souls, wind that can carry away
  buildings, or any other weather you can imagine: ask and it shall
  come.
\end{aspell}

\begin{aspell}{Repair}{}
  Choose one event in the target’s past. All effects of that event,
  including damage, poison, disease, and magical effects, are ended
  and repaired. HP and diseases are healed, poisons are neutralized,
  magical effects are ended.
\end{aspell}

\begin{aspell}{Divine Presence}{Ongoing}
  Every creature must ask your leave to enter your presence, and you
  must give permission aloud for them to enter. Any creature without
  your leave takes an extra 1d10 damage whenever they take damage in
  your presence. While this spell is ongoing you take -1 to cast a
  spell.
\end{aspell}

\vfill\null
\columnbreak

\begin{aspell}{Consume Unlife}{}
  The mindless undead creature you touch is destroyed and you steal
  its death energy to heal yourself or the next ally you touch. The
  amount of damage healed is equal to the HP that the creature had
  remaining before you destroyed it.
\end{aspell}

\begin{aspell}{Plague}{Ongoing}
  Name a city, town, encampment, or other place where people live. As
  long as this spell is active that place is beset by a plague
  appropriate to your deity’s domains (locusts, death of the first
  born, etc.) While this spell is ongoing you take -1 to cast a spell.
\end{aspell}
\vfill\null
\end{multicols}

\end{document}
