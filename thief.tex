\documentclass[8pt]{extarticle}
\usepackage[dvipsnames]{xcolor}
\usepackage{hyperref}
\usepackage{lmodern}
\usepackage{amssymb,amsmath}
\usepackage{ifxetex,ifluatex}
\usepackage{anyfontsize}
\usepackage[percent]{overpic}
\usepackage[margin=0.4in]{geometry}
\usepackage{multicol}
\setlength{\columnsep}{0.05cm}
\usepackage[T1]{fontenc}
\usepackage[utf8]{inputenc}
\usepackage{fontspec} % For loading fonts
\usepackage{titlesec}

\setmainfont{PT Serif}
\newfontfamily\headingfont[]{Metamorphous}
\titleformat*{\section}{\LARGE\headingfont}
\titleformat*{\subsection}{\Large\headingfont}

\newenvironment{amove}[1]
{\Checkbox{6pt}\ {\color{MoveBlue}$\diamond$\headingfont #1}\begin{quote}
}
{\end{quote}
}

\newenvironment{optfeature}[2][]
{\Checkbox{6pt}\ {\headingfont #2}\hfill\textit{#1}\phantom{asdf}\begin{quote}
}
{\end{quote}
}

\newenvironment{aspell}[2]
{\Checkbox{6pt}\ {\headingfont\ \spell{#1}}\hfill\textit{#2}\phantom{asdf}\begin{quote}
}
{\end{quote}
}

\newenvironment{fragment}[1]
{\begin{quote}{\headingfont #1}\begin{quote}
}
{\end{quote}\end{quote}
}

\newcommand{\subheader}[1]{\large\headingfont #1}

\newenvironment{basicmove}[1]
{\begin{quote}{\color{MoveBlue}$\diamond$\headingfont #1}\begin{quote}
}
{\end{quote}\end{quote}
}

\makeatletter

\newcommand{\choicelabel}[1]{
  {\hss\llap{\Checkbox{6pt}}}
}
\newcommand{\choicelabeldef}{
  \@gobble{choicelabeldef}
}

\newenvironment{choices}
{
  \itemize
  \let\makelabel\choicelabel
  \let\@itemlabel\choicelabeldef
}
{\enditemize
}
\makeatother

\newcommand{\choice}{\Checkbox{6pt} }

\pagestyle{empty}
\IfFileExists{upquote.sty}{\usepackage{upquote}}{}
% use microtype if available
\IfFileExists{microtype.sty}{%
\usepackage[]{microtype}
\UseMicrotypeSet[protrusion]{basicmath} % disable protrusion for tt fonts
}{}
\PassOptionsToPackage{hyphens}{url} % url is loaded by hyperref

\makeatother
% Scale images if necessary, so that they will not overflow the page
% margins by default, and it is still possible to overwrite the defaults
% using explicit options in \includegraphics[width, height, ...]{}
\setkeys{Gin}{width=\maxwidth,height=\maxheight,keepaspectratio}
\IfFileExists{parskip.sty}{%
\usepackage{parskip}
}{% else
\setlength{\parindent}{0pt}
\setlength{\parskip}{6pt plus 2pt minus 1pt}
}
\setlength{\emergencystretch}{3em}  % prevent overfull lines
\providecommand{\tightlist}{%
  \setlength{\itemsep}{0pt}\setlength{\parskip}{0pt}}
\setcounter{secnumdepth}{0}
% Redefines (sub)paragraphs to behave more like sections
\ifx\paragraph\undefined\else
\let\oldparagraph\paragraph
\renewcommand{\paragraph}[1]{\oldparagraph{#1}\mbox{}}
\fi
\ifx\subparagraph\undefined\else
\let\oldsubparagraph\subparagraph
\renewcommand{\subparagraph}[1]{\oldsubparagraph{#1}\mbox{}}
\fi

% set default figure placement to htbp
\makeatletter
\def\fps@figure{htbp}
\makeatother

\setlength{\multicolsep}{6.0pt plus 2.0pt minus 1.5pt}% 50% of original values

\date{}

\usepackage{etoolbox}
\patchcmd{\quote}{\rightmargin}{\leftmargin 1em \rightmargin}{}{}

\usepackage{tikz}
\newcommand{\Checkbox}[1]{\tikz{\path[draw=black] (0,0) rectangle (#1,#1);}}

\newcommand{\pbClass}[1]{\newcommand{\Class}{#1}}
\newcommand{\pbBaseHP}[1]{\newcommand{\BaseHP}{#1}}
\newcommand{\pbDamage}[1]{\newcommand{\Damage}{#1}}
\newcommand{\Look}{}
\newcommand{\Names}{}
\makeatletter
\newcommand{\pbLook}[1]{\g@addto@macro\Look{\par#1}}
\newcommand{\pbNames}[2]{\g@addto@macro\Names{\par\hangindent=0.2in#1: #2}}
\makeatother

\newcommand{\leftbanner}[1]{
  \begin{overpic}[width=3.1in,height=0.45in]{assets/short_left.png}
\put (2,4) {\fontsize{16}{40}\selectfont \textcolor{white}{\headingfont #1}}
\end{overpic}
}

\newcommand{\rightbanner}[1]{
  \begin{overpic}[width=4.4in,height=0.45in]{assets/long_right.png}
\put (5,4) {\fontsize{16}{40}\selectfont \textcolor{white}{\headingfont #1}}
\end{overpic}
}

\newcommand{\gearbanner}{
\begin{overpic}[width=7.47986in,height=0.40945in]{assets/templateEquip.png}
\put (3,2) {\fontsize{16}{40}\selectfont \textcolor{white}{\headingfont Gear}}
\end{overpic}
}

\newcommand{\topbanner}[1]{
  \begin{overpic}[width=7.47986in,height=1.0in]{assets/templateRuleHeader.png}
  \put (1,2) {\fontsize{32}{40}\selectfont\headingfont \textcolor{white}{#1}}
\end{overpic}
}

\newcommand{\widebanner}[1]{
  \begin{overpic}[width=7.47986in,height=1.0in]{assets/templateThinHeader.png}
  \put (1,1) {\fontsize{16}{40}\selectfont\headingfont \textcolor{white}{#1}}
\end{overpic}
}

\newcommand{\charbanner}{
  \begin{overpic}[width=8.008in,height=3.0in]{assets/charsheet_upper.png}
  % names
  \put(1, 30) {\parbox{4.3in}{\fontsize{12}{12}\Names}}
  % look
  \put(59, 30) {\parbox{3in}{\fontsize{12}{12}\Look}}

  % some stats: damage...
  \put (25,4) {\makebox[0pt]{\fontsize{18}{10}\selectfont\headingfont \textcolor{black}{D\Damage{}}}}
  % max HP...
  \put (89,6) {\fontsize{6}{8}\selectfont\headingfont \textcolor{white}{Your max HP is}}
  % and Constitution
  \put (89,4.6) {\fontsize{6}{8}\selectfont\headingfont \textcolor{white}{\BaseHP{} + Constitution}}
\end{overpic}
}

\newcommand{\charlower}{
  \vfill\null
  \begin{overpic}[width=7.47986in,height=1.0in]{assets/charsheet_lower.png}
  \put (10,1) {\fontsize{32}{40}\selectfont\headingfont \textcolor{white}{The \Class}}
\end{overpic}
}

\definecolor{CondRed}{RGB}{153,51,51}
\definecolor{MoveBlue}{RGB}{51,102,153}
\definecolor{SpellPurp}{RGB}{153,51,102}
\definecolor{TagGreen}{RGB}{51,153,102}

\newcommand{\condition}[1]{\textbf{\color{CondRed} #1}}
\newcommand{\move}[1]{{\color{MoveBlue}$\diamond$#1}}
\newcommand{\spell}[1]{{\color{SpellPurp}$\star$#1}}
\newcommand{\itag}[1]{\textit{\color{TagGreen}#1}}
\newcommand{\ntag}[2]{\textit{\color{TagGreen}#1 #2}}

% specific tags
\newcommand{\weight}[1]{\ntag{#1}{weight}}
\newcommand{\damage}[1]{\ntag{#1}{damage}}
\newcommand{\armor}[1]{\ntag{#1}{armor}}
\newcommand{\armorForward}[1]{\ntag{#1}{armor} forward}
\newcommand{\uses}[1]{\ntag{#1}{uses}}
\newcommand{\ammo}[1]{\ntag{#1}{ammo}}

\newcommand{\hexes}[1]{\textit{#1 hexes}}
\newcommand{\forward}[1]{#1 forward}
\newcommand{\ongoing}[1]{#1 ongoing}
\newcommand{\yourLoad}[1]{Your Load is \textbf{#1+STR}}

\newcommand{\onSuccess}{\textbf{On a 10+}}
\newcommand{\onPlainSuccess}{\textbf{On a 10--11}}
\newcommand{\onMassiveSuccess}{\textbf{On a 12+}}
\newcommand{\onPartial}{\textbf{On a 7--9}}
\newcommand{\onHit}{\textbf{On a 7+}}
\newcommand{\onMiss}{\textbf{On a miss}}

\newcommand{\moveReplaces}[1]{\textbf{Replaces}: \move{#1}}
\newcommand{\moveRequires}[1]{\textbf{Requires}: \move{#1}}
\newcommand{\moveRequiresLst}[1]{\textbf{Requires}: #1}

\newcommand{\onMassiveSuccessFor}[1]{%
  When you use \move{#1} and \textbf{roll a 12+}}
\newcommand{\onPlainSuccessFor}[1]{%
  When you use \move{#1} and \textbf{roll a 10--11}}
\newcommand{\onPartialFor}[1]{%
  When you use \move{#1} and \textbf{roll a 7+}}

\newcommand{\advancesigil}{$\triangleright$}
\newcommand{\firstAdvances}{\advancesigil When you \textbf{gain a
    level from 2--5}, choose from these moves.}
\newcommand{\secondAdvances}{\advancesigil When you \textbf{gain a
    level from 6--10}, choose from these moves or the level 2--5
  moves.}

\newcommand{\blank}{\underline{\phantom{mountain}}}
\newcommand{\directive}[1]{\textbf{#1}}

\openup -0.2em

\pbClass{Thief}
\pbBaseHP{6}
\pbDamage{8}

\pbNames{Halfling}{Felix, Rook, Mouse, Sketch, Trixie, Robin, Omar,
  Brynn, Bug}
\pbNames{Human}{Sparrow, Shank, Jack, Marlow, Dodge, Rat, Pox, Humble,
  Farley}

\pbLook{Shifty Eyes or Criminal Eyes}
\pbLook{Hooded Head, Messy Hair, or Cropped Hair}
\pbLook{Dark Clothes, Fancy Clothes, or Common Clothes}
\pbLook{Lithe Body, Knobby Body, or Flabby Body}

\begin{document}
\charbanner

\begin{minipage}[t]{3.2in}
\leftbanner{Folk}

\begin{optfeature}{Halfling}
  When you \condition{attack with a ranged weapon}, deal +2 damage.
\end{optfeature}

\begin{optfeature}{Human}
  You are a professional. When you \move{Spout Lore} or
  \move{Discern Realities} about criminal activities, take +1.
\end{optfeature}

\ 

\leftbanner{Alignment}

\begin{optfeature}{Chaotic}
  Leap into danger without a plan.
\end{optfeature}

\begin{optfeature}{Neutral}
  Avoid detection or infiltrate a location.
\end{optfeature}

\begin{optfeature}{Evil}
  Shift danger or blame from yourself to someone else.
\end{optfeature}

\ 

\leftbanner{Bonds}


\vfill\null
\end{minipage}
\begin{minipage}[t]{4.6in}


\rightbanner{Starting Moves}

\begin{basicmove}{Trap Expert}
  When you \condition{spend a moment to survey a dangerous area}, roll
  +DEX. \onSuccess, hold 3. \onPartial, hold 1. Spend your hold as you
  walk through the area to ask these questions:

  \begin{itemize}
  \item Is there a trap here and if so, what activates it?
  \item What does the trap do when activated?
  \item What else is hidden here?
  \end{itemize}
\end{basicmove}
\

\begin{basicmove}{Tricks of the Trade}
  When you \condition{pick locks or pockets or disable traps}, roll
  +DEX. \onSuccess, you do it, no problem. \onPartial, you still do
  it, but the GM will offer you two options between suspicion, danger,
  or cost.
\end{basicmove}
\

\begin{basicmove}{Backstab}
  When you \condition{attack a surprised or defenseless enemy with a
    melee weapon}, you can choose to deal your damage or
  roll +DEX. \onSuccess, choose two. \onPartial, choose one.

  \begin{itemize}
  \item You don’t get into melee with them
  \item You deal your damage+1d6
  \item You create an advantage, \forward{+1} to you or an ally acting
    on it
  \item Reduce their armor by 1 until they repair it
  \end{itemize}

\end{basicmove}
\


\begin{basicmove}{Flexible Morals}
  When \condition{someone tries to detect your alignment} you can tell
  them any alignment you like.
\end{basicmove}
\

\begin{basicmove}{Poisoner}
  You’ve mastered the care and use of a poison. Choose a poison from
  the list below; that poison is no longer dangerous for you to
  use. You also start with three uses of the poison you
  choose. Whenever you have time to gather materials and a safe place
  to brew you can make three uses of the poison you choose for
  free. Note that some poisons are applied, meaning you have to
  carefully apply it to the target or something they eat or
  drink. Touch poisons just need to touch the target, they can even be
  used on the blade of a weapon.

  \begin{itemize}
  \item Oil of Tagit (\itag{applied}): The target falls into a light
    sleep
  \item Bloodweed (\itag{touch}): The target deals -1d4 damage ongoing
    until cured
  \item Goldenroot (\itag{applied}): The target treats the next
    creature they see as a trusted ally, until proved otherwise
  \item Serpent’s Tears (\itag{touch}): Anyone dealing damage to the
    target rolls twice and takes the better result.
  \end{itemize}
\end{basicmove}


\vfill\null
\end{minipage}

\charlower

\clearpage

\gearbanner

\begin{multicols}{2}
  \yourLoad{9}. You start with dungeon rations (\uses{5}, \weight{1}),
  leather armor (\armor{1}, \weight{1}), 3 uses of your chosen poison,
  and 10 coins. Choose your arms:

  \begin{choices}
  \item Dagger (\itag{hand}, \weight{1}) and short sword
    (\itag{close}, \weight{1})
  \item Rapier(\itag{close}, \itag{precise}, \weight{1})
  \end{choices}

  Choose a ranged weapon:

  \begin{choices}
  \item 3 throwing daggers (\itag{thrown}, \itag{near}, \weight{0})
  \item Ragged Bow (\itag{near}, \weight{2}) and bundle of arrows
    (\ammo{3}, \weight{1})
  \end{choices}

  Choose one:

  \begin{choices}
  \item Adventuring gear (\weight{1})
  \item Healing potion (\weight{0})
  \end{choices}

\columnbreak

\

\end{multicols}

\widebanner{Advanced Moves}

\begin{multicols}{2}
  \firstAdvances

  \begin{amove}{Cheap Shot}
    When using a \itag{precise} or \itag{hand} weapon, your
    \move{Backstab} deals an extra +1d6 damage.
  \end{amove}

  \begin{amove}{Cautious}
    When you use \move{Trap Expert} you always get +1 hold, even on a
    6-.
  \end{amove}

  \begin{amove}{Wealth and Taste}
    When you make a show of flashing around your most valuable
    possession, choose someone present. They will do anything they can
    to obtain your item or one like it.
  \end{amove}

  \begin{amove}{Shoot First}
    You’re never caught by surprise. When \condition{an enemy would
      get the drop on you}, you get to act first instead.
  \end{amove}

  \begin{amove}{Poison Master}
    After you’ve used a poison once it’s no longer dangerous for you
    to use.
  \end{amove}

  \begin{amove}{Envenom}
    You can apply even complex poisons with a pinprick. When you
    \condition{apply a poison that’s not dangerous for you to use to
      your weapon} it’s \itag{touch} instead of \itag{applied}.
  \end{amove}

  \begin{amove}{Brewer}
    When you have time to gather materials and a safe place to brew
    you can create three doses of any one poison you’ve used before.
  \end{amove}

  \begin{amove}{Underdog}
    When you’re \condition{outnumbered}, you have +1 armor.
  \end{amove}

  \begin{amove}{Connections}
    When you \condition{put out word to the criminal underbelly about
      something you want or need}, roll +CHA. \onSuccess, someone has
    it, just for you. \onPartial, you’ll have to settle for something
    close or it comes with strings attached, your call.
  \end{amove}

\secondAdvances

\begin{amove}{Dirty Fighter}
\moveReplaces{Cheap Shot}

  When using a \itag{precise} or \itag{hand} weapon, your
  \move{Backstab} deals an extra +1d8 damage and all other attacks
  deal +1d4 damage.
\end{amove}

\begin{amove}{Extremely Cautious}
\moveReplaces{Cautious}

  When you use \move{Trap Expert} you always get +1 hold, even on a
  6-. \onMassiveSuccess, you get 3 hold and the next time you come
  near a trap the GM will immediately tell you what it does, what
  triggers it, who set it, and how you can use it to your advantage.
\end{amove}

\begin{amove}{Alchemist}
\moveReplaces{Brewer}

  When you have you have time to gather materials and a safe place to
  brew you can create three doses of any poison you’ve used
  before. Alternately you can describe the effects of a poison you’d
  like to create. The GM will tell you that you can create it, but
  with one or more caveats:

  \begin{itemize}
  \item It will only work under specific circumstances
  \item The best you can manage is a weaker version
  \item It’ll take a while to take effect
  \item It’ll have obvious side effects
  \end{itemize}

\end{amove}

\begin{amove}{Serious Underdog}
\moveReplaces{Underdog}

  You have \armor{+1}. When you’re outnumbered, you have \armor{+2}
  instead.
\end{amove}

\begin{amove}{Evasion}
  \onMassiveSuccessFor{Defy Danger}, you transcend the danger. You not
  only do what you set out to, but the GM will offer you a better
  outcome, true beauty, or a moment of grace.
\end{amove}

\begin{amove}{Strong Arm, True Aim}
  You can throw any melee weapon, using it to \move{Volley}. A thrown
  melee weapon is gone; you can never choose to reduce ammo on a 7–9.
\end{amove}

\begin{amove}{Escape Route}
  When you’re \condition{in too deep and need a way out}, name your
  escape route and roll +DEX. \onSuccess, you’re gone. \onPartial, you
  can stay or go, but if you go it costs you: leave something behind
  or take something with you, the GM will tell you what.
\end{amove}

\begin{amove}{Disguise}
  When you have time and materials you can create a disguise that will
  fool anyone into thinking you’re another creature of about the same
  size and shape. Your actions can give you away but your appearance
  won’t.
\end{amove}

\begin{amove}{Heist}
  When you \condition{take time to make a plan to steal something},
  name the thing you want to steal and ask the GM these
  questions. When acting on the answers you and your allies take
  \forward{+1}.

  \begin{itemize}
  \item Who will notice it’s missing?
  \item What’s its most powerful defense?
  \item Who will come after it?
  \item Who else wants it?
  \end{itemize}
\end{amove}

\vfill\null
\end{multicols}

\end{document}
