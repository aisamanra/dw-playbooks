\documentclass[8pt]{extarticle}
\usepackage[dvipsnames]{xcolor}
\usepackage{hyperref}
\usepackage{lmodern}
\usepackage{amssymb,amsmath}
\usepackage{ifxetex,ifluatex}
\usepackage{anyfontsize}
\usepackage[percent]{overpic}
\usepackage[margin=0.4in]{geometry}
\usepackage{multicol}
\setlength{\columnsep}{0.05cm}
\usepackage[T1]{fontenc}
\usepackage[utf8]{inputenc}
\usepackage{fontspec} % For loading fonts
\usepackage{titlesec}

\setmainfont{PT Serif}
\newfontfamily\headingfont[]{Metamorphous}
\titleformat*{\section}{\LARGE\headingfont}
\titleformat*{\subsection}{\Large\headingfont}

\newenvironment{amove}[1]
{\Checkbox{6pt}\ {\color{MoveBlue}$\diamond$\headingfont #1}\begin{quote}
}
{\end{quote}
}

\newenvironment{optfeature}[2][]
{\Checkbox{6pt}\ {\headingfont #2}\hfill\textit{#1}\phantom{asdf}\begin{quote}
}
{\end{quote}
}

\newenvironment{aspell}[2]
{\Checkbox{6pt}\ {\headingfont\ \spell{#1}}\hfill\textit{#2}\phantom{asdf}\begin{quote}
}
{\end{quote}
}

\newenvironment{fragment}[1]
{\begin{quote}{\headingfont #1}\begin{quote}
}
{\end{quote}\end{quote}
}

\newcommand{\subheader}[1]{\large\headingfont #1}

\newenvironment{basicmove}[1]
{\begin{quote}{\color{MoveBlue}$\diamond$\headingfont #1}\begin{quote}
}
{\end{quote}\end{quote}
}

\makeatletter

\newcommand{\choicelabel}[1]{
  {\hss\llap{\Checkbox{6pt}}}
}
\newcommand{\choicelabeldef}{
  \@gobble{choicelabeldef}
}

\newenvironment{choices}
{
  \itemize
  \let\makelabel\choicelabel
  \let\@itemlabel\choicelabeldef
}
{\enditemize
}
\makeatother

\newcommand{\choice}{\Checkbox{6pt} }

\pagestyle{empty}
\IfFileExists{upquote.sty}{\usepackage{upquote}}{}
% use microtype if available
\IfFileExists{microtype.sty}{%
\usepackage[]{microtype}
\UseMicrotypeSet[protrusion]{basicmath} % disable protrusion for tt fonts
}{}
\PassOptionsToPackage{hyphens}{url} % url is loaded by hyperref

\makeatother
% Scale images if necessary, so that they will not overflow the page
% margins by default, and it is still possible to overwrite the defaults
% using explicit options in \includegraphics[width, height, ...]{}
\setkeys{Gin}{width=\maxwidth,height=\maxheight,keepaspectratio}
\IfFileExists{parskip.sty}{%
\usepackage{parskip}
}{% else
\setlength{\parindent}{0pt}
\setlength{\parskip}{6pt plus 2pt minus 1pt}
}
\setlength{\emergencystretch}{3em}  % prevent overfull lines
\providecommand{\tightlist}{%
  \setlength{\itemsep}{0pt}\setlength{\parskip}{0pt}}
\setcounter{secnumdepth}{0}
% Redefines (sub)paragraphs to behave more like sections
\ifx\paragraph\undefined\else
\let\oldparagraph\paragraph
\renewcommand{\paragraph}[1]{\oldparagraph{#1}\mbox{}}
\fi
\ifx\subparagraph\undefined\else
\let\oldsubparagraph\subparagraph
\renewcommand{\subparagraph}[1]{\oldsubparagraph{#1}\mbox{}}
\fi

% set default figure placement to htbp
\makeatletter
\def\fps@figure{htbp}
\makeatother

\setlength{\multicolsep}{6.0pt plus 2.0pt minus 1.5pt}% 50% of original values

\date{}

\usepackage{etoolbox}
\patchcmd{\quote}{\rightmargin}{\leftmargin 1em \rightmargin}{}{}

\usepackage{tikz}
\newcommand{\Checkbox}[1]{\tikz{\path[draw=black] (0,0) rectangle (#1,#1);}}

\newcommand{\pbClass}[1]{\newcommand{\Class}{#1}}
\newcommand{\pbBaseHP}[1]{\newcommand{\BaseHP}{#1}}
\newcommand{\pbDamage}[1]{\newcommand{\Damage}{#1}}
\newcommand{\Look}{}
\newcommand{\Names}{}
\makeatletter
\newcommand{\pbLook}[1]{\g@addto@macro\Look{\par#1}}
\newcommand{\pbNames}[2]{\g@addto@macro\Names{\par\hangindent=0.2in#1: #2}}
\makeatother

\newcommand{\leftbanner}[1]{
  \begin{overpic}[width=3.1in,height=0.45in]{assets/short_left.png}
\put (2,4) {\fontsize{16}{40}\selectfont \textcolor{white}{\headingfont #1}}
\end{overpic}
}

\newcommand{\rightbanner}[1]{
  \begin{overpic}[width=4.4in,height=0.45in]{assets/long_right.png}
\put (5,4) {\fontsize{16}{40}\selectfont \textcolor{white}{\headingfont #1}}
\end{overpic}
}

\newcommand{\gearbanner}{
\begin{overpic}[width=7.47986in,height=0.40945in]{assets/templateEquip.png}
\put (3,2) {\fontsize{16}{40}\selectfont \textcolor{white}{\headingfont Gear}}
\end{overpic}
}

\newcommand{\topbanner}[1]{
  \begin{overpic}[width=7.47986in,height=1.0in]{assets/templateRuleHeader.png}
  \put (1,2) {\fontsize{32}{40}\selectfont\headingfont \textcolor{white}{#1}}
\end{overpic}
}

\newcommand{\widebanner}[1]{
  \begin{overpic}[width=7.47986in,height=1.0in]{assets/templateThinHeader.png}
  \put (1,1) {\fontsize{16}{40}\selectfont\headingfont \textcolor{white}{#1}}
\end{overpic}
}

\newcommand{\charbanner}{
  \begin{overpic}[width=8.008in,height=3.0in]{assets/charsheet_upper.png}
  % names
  \put(1, 30) {\parbox{4.3in}{\fontsize{12}{12}\Names}}
  % look
  \put(59, 30) {\parbox{3in}{\fontsize{12}{12}\Look}}

  % some stats: damage...
  \put (25,4) {\makebox[0pt]{\fontsize{18}{10}\selectfont\headingfont \textcolor{black}{D\Damage{}}}}
  % max HP...
  \put (89,6) {\fontsize{6}{8}\selectfont\headingfont \textcolor{white}{Your max HP is}}
  % and Constitution
  \put (89,4.6) {\fontsize{6}{8}\selectfont\headingfont \textcolor{white}{\BaseHP{} + Constitution}}
\end{overpic}
}

\newcommand{\charlower}{
  \vfill\null
  \begin{overpic}[width=7.47986in,height=1.0in]{assets/charsheet_lower.png}
  \put (10,1) {\fontsize{32}{40}\selectfont\headingfont \textcolor{white}{The \Class}}
\end{overpic}
}

\definecolor{CondRed}{RGB}{153,51,51}
\definecolor{MoveBlue}{RGB}{51,102,153}
\definecolor{SpellPurp}{RGB}{153,51,102}
\definecolor{TagGreen}{RGB}{51,153,102}

\newcommand{\condition}[1]{\textbf{\color{CondRed} #1}}
\newcommand{\move}[1]{{\color{MoveBlue}$\diamond$#1}}
\newcommand{\spell}[1]{{\color{SpellPurp}$\star$#1}}
\newcommand{\itag}[1]{\textit{\color{TagGreen}#1}}
\newcommand{\ntag}[2]{\textit{\color{TagGreen}#1 #2}}

% specific tags
\newcommand{\weight}[1]{\ntag{#1}{weight}}
\newcommand{\damage}[1]{\ntag{#1}{damage}}
\newcommand{\armor}[1]{\ntag{#1}{armor}}
\newcommand{\armorForward}[1]{\ntag{#1}{armor} forward}
\newcommand{\uses}[1]{\ntag{#1}{uses}}
\newcommand{\ammo}[1]{\ntag{#1}{ammo}}

\newcommand{\hexes}[1]{\textit{#1 hexes}}
\newcommand{\forward}[1]{#1 forward}
\newcommand{\ongoing}[1]{#1 ongoing}
\newcommand{\yourLoad}[1]{Your Load is \textbf{#1+STR}}

\newcommand{\onSuccess}{\textbf{On a 10+}}
\newcommand{\onPlainSuccess}{\textbf{On a 10--11}}
\newcommand{\onMassiveSuccess}{\textbf{On a 12+}}
\newcommand{\onPartial}{\textbf{On a 7--9}}
\newcommand{\onHit}{\textbf{On a 7+}}
\newcommand{\onMiss}{\textbf{On a miss}}

\newcommand{\moveReplaces}[1]{\textbf{Replaces}: \move{#1}}
\newcommand{\moveRequires}[1]{\textbf{Requires}: \move{#1}}
\newcommand{\moveRequiresLst}[1]{\textbf{Requires}: #1}

\newcommand{\onMassiveSuccessFor}[1]{%
  When you use \move{#1} and \textbf{roll a 12+}}
\newcommand{\onPlainSuccessFor}[1]{%
  When you use \move{#1} and \textbf{roll a 10--11}}
\newcommand{\onPartialFor}[1]{%
  When you use \move{#1} and \textbf{roll a 7+}}

\newcommand{\advancesigil}{$\triangleright$}
\newcommand{\firstAdvances}{\advancesigil When you \textbf{gain a
    level from 2--5}, choose from these moves.}
\newcommand{\secondAdvances}{\advancesigil When you \textbf{gain a
    level from 6--10}, choose from these moves or the level 2--5
  moves.}

\newcommand{\blank}{\underline{\phantom{mountain}}}
\newcommand{\directive}[1]{\textbf{#1}}

\openup -0.2em

\pbClass{Ranger}
\pbBaseHP{8}
\pbDamage{8}

\pbNames{Elf}{Throndir, Elrosine, Aranwe, Celion, Dambrath, Lanethe}
\pbNames{Human}{Jonah, Halek, Brandon, Emory, Shrike, Nora, Diana}

\pbLook{Wild Eyes, Sharp Eyes, or Animal Eyes}
\pbLook{Hooded Head, Wild Hair, or Bald}
\pbLook{Cape, Camouflage, or Traveling Clothes}
\pbLook{Lithe Body, Wild Body, or Sharp Body}

\begin{document}
\openup -0.2em

\charbanner

\begin{minipage}[t]{3.2in}
\leftbanner{Folk}

\begin{optfeature}{Elf}
  When you \move{Undertake a Perilous Journey} through wilderness,
  whatever job you take you succeed as if you rolled a 10+.
\end{optfeature}

\begin{optfeature}{Human}
  When you make camp in a dungeon or city, you don’t need to consume a
  ration.
\end{optfeature}

\ 

\leftbanner{Alignment}

\begin{optfeature}{Chaotic}
  Free someone from literal or figurative bonds.
\end{optfeature}

\begin{optfeature}{Good}
  Endanger yourself to combat an unnatural threat.
\end{optfeature}

\begin{optfeature}{Neutral}
  Help an animal or spirit of the wild.
\end{optfeature}

\ 

\leftbanner{Bonds}

\vfill\null
\end{minipage}
\begin{minipage}[t]{4.6in}


\rightbanner{Starting Moves}

\begin{basicmove}{Hunt and Track (CHA)}
  When you \condition{follow a trail of clues left behind by passing
    creatures}, roll +WIS. \onHit, you follow the creature's
  trail until there's a significant change in its direction or mode of
  travel. \onSuccess, you also choose 1:
  \begin{itemize}
  \item Gain a useful bit of information about your quarry, the GM
    will tell you what
  \item Determine what caused the trail to end
  \end{itemize}
\end{basicmove}
\

\begin{basicmove}{Called Shot}
  When you \condition{attack a defenseless or surprised enemy at range},
  you can choose to deal your damage or name your target and roll
  +DEX.

  \begin{description}
  \item[Head] \onPartial, they do nothing but stand and drool for a
    few moments. \onSuccess, as 7--9, plus your damage.
  \item[Arms] \onPartial, they drop anything they're
    holding. \onSuccess, as 7--9, plus your damage.
  \item[Legs] \onPartial, they're hobbled and slow-moving. \onSuccess,
    as 7--9, plus your damage.
  \end{description}
\end{basicmove}
\

\begin{basicmove}{Animal Companion}
  You have a supernatural connection with a loyal animal. You can't
  talk to it per se, but it always acts as you wish it to. Name your
  animal companion and choose a species:

  \begin{quote}
    \textit{wolf, cougar, bear, eagle, dog, hawk, cat, owl, pigeon,
      rat, mule}
  \end{quote}

  Choose a base:
  \begin{choices}
  \item Ferocity +1, Cunning +1, 1 Armor, Instinct +1
  \item Ferocity +2, Cunning +2, 0 Armor, Instinct +1
  \item Ferocity +1, Cunning +2, 1 Armor, Instinct +1
  \item Ferocity +3, Cunning +1, 1 Armor, Instinct +2
  \end{choices}

  Choose as many strengths as its ferocity:

  \begin{quote}
    \textit{fast, burly, huge, calm, adaptable, quick reflexes,
      tireless, camouflage, ferocious, intimidating, keen senses,
      stealthy}
  \end{quote}

  Your animal companion is trained to fight humanoids. Choose as many
  additional trainings as its cunning:

  \begin{quote}
    \textit{hunt, search, scout, guard, fight monsters, perform,
      labor, travel}
  \end{quote}

  Choose as many weaknesses as its instinct:

  \begin{quote}
    \textit{flighty, savage, slow, broken, frightening, forgetful,
      stubborn, lame}
  \end{quote}
\end{basicmove}
\

\begin{basicmove}{Command}
  When you \condition{work with your animal companion on something
    it's trained in}:

  \begin{itemize}
  \item ...and you attack the same target, add its ferocity to your damage
  \item ...and you track, add its cunning to your roll
  \item ...and you take damage, add its armor to your armor
  \item ...and you discern realities, add its cunning to your roll
  \item ...and you parley, add its cunning to your roll
  \item ...and someone interferes with you, add its instinct to their roll
  \end{itemize}
\end{basicmove}


\vfill\null
\end{minipage}
\charlower
\clearpage

\gearbanner

\begin{multicols}{2}

  \begin{quote}

    \yourLoad{11}. You start with dungeon rations (\uses{5},
    \weight{1}), leather armor (\armor{1}, \weight{1}), and a bundle of
    arrows (\ammo{3}, \weight{1}). Choose your armament:

    \begin{choices}
    \item Hunter’s bow (\itag{near}, \itag{far}, \weight{1}) and short
      sword (\itag{close}, \weight{1})
    \item Hunter’s bow (\itag{near}, \itag{far}, \weight{1}) and spear
      (\itag{reach}, \weight{1})
    \end{choices}

    Choose one:
    \begin{choices}
    \item Adventuring gear (\weight{1}) and dungeon rations (\weight{1})
    \item Adventuring gear (\weight{1}) and bundle of arrows
      (\ammo{3}, \weight{1})
    \end{choices}


  \end{quote}

\ 

\columnbreak

\

\end{multicols}

\widebanner{Advanced Moves}

\begin{multicols}{2}
  $\triangleright$ \textbf{You may take this feature only if it is
    your first advancement}.

  \begin{optfeature}{Half-Elven}
    Somewhere in your lineage lies mixed blood and it begins to show
    its presence. You gain the elf starting move if you took the human
    one at character creation, or vice versa.
  \end{optfeature}

  \

  \firstAdvances

\begin{amove}{Wild Empathy}
  You can speak with and understand animals.
\end{amove}

\begin{amove}{Familiar Prey}
  When you \move{Spout Lore} about a monster, you use WIS instead of
  INT.
\end{amove}

\begin{amove}{Viper’s Strike}
  When you \condition{strike an enemy with two weapons at once}, add
  an extra 1d4 damage for your off-hand strike.
\end{amove}

\begin{amove}{Camouflage}
  When you \condition{keep still in natural surroundings}, enemies
  never spot you until you make a movement.
\end{amove}

\begin{amove}{Man’s Best Friend}
  When you \condition{allow your animal companion to take a blow that
    was meant for you}, the damage is negated and your animal
  companion’s ferocity becomes 0. If its ferocity is already 0 you
  can’t use this ability. When you have a few hours of rest with your
  animal companion its ferocity returns to normal.
\end{amove}

\begin{amove}{Blot Out the Sun}
  When you \move{Volley} you may spend extra ammo before rolling. For
  each point of ammo spent you may choose an extra target. Roll once
  and apply damage to all targets.
\end{amove}

\begin{amove}{Well-Trained}
  Choose another training for your animal companion.
\end{amove}

\begin{amove}{God Amidst the Wastes}
  Dedicate yourself to a deity (name a new one or choose one that’s
  already been established). You gain the \move{Commune} and
  \move{Cast a Spell} Cleric moves. When you select this move, treat
  yourself as a Cleric of level 1 for using spells. Every time you
  gain a level thereafter, increase your effective Cleric level by 1.
\end{amove}

\begin{amove}{Follow Me}
  When you \move{Undertake a Perilous Journey} you can take two
  roles. You make a separate roll for each.
\end{amove}

\begin{amove}{A Safe Place}
  When you \condition{set the watch for the night}, everyone takes +1
  to \move{Take Watch}.
\end{amove}


\vfill\null
\columnbreak

\secondAdvances

\begin{amove}{Wild Speech}
\moveReplaces{Wild Empathy}

  You can speak with and understand any non-magical, non-planar
  creature.
\end{amove}

\begin{amove}{Hunter’s Prey}
\moveReplaces{Familiar Prey}

  When you \move{Spout Lore} about a monster you use WIS instead of
  INT. \onMassiveSuccess, in addition to the normal effects, you get
  to ask the GM any one question about the subject.
\end{amove}

\begin{amove}{Viper’s Fangs}
\moveReplaces{Viper’s Strike}

  When you \condition{strike an enemy with two weapons at once}, add
  an extra 1d8 damage for your off-hand strike.
\end{amove}

\begin{amove}{Smaug’s Belly}
  When you know your target’s weakest point your arrows have
  \ntag{2}{piercing}.
\end{amove}

\begin{amove}{Strider}
\moveReplaces{Follow Me}

When you \move{Undertake a Perilous Journey} you can take two
roles. Roll twice and use the better result for both roles.
\end{amove}

\begin{amove}{A Safer Place}
\moveReplaces{A Safe Place}

  When you \condition{set the watch for the night} everyone takes +1
  to take watch. After a night in camp when you set the watch everyone
  takes \forward{+1}.
\end{amove}

\begin{amove}{Observant}
  When you \move{Hunt and Track}, on a hit you may also ask one
  question about the creature you are tracking from the \move{Discern
    Realities} list for free.
\end{amove}

\begin{amove}{Special Trick}
  Choose a move from another class. So long as you are working with
  your animal companion you have access to that move.
\end{amove}

\begin{amove}{Unnatural Ally}
  Your animal companion is a monster, not an animal. Describe it. Give
  it +2 ferocity and +1 instinct, plus a new training.
\end{amove}


\vfill\null
\end{multicols}

\end{document}
